\documentclass[Draft.tex]{subfiles}
\begin{document}
\chapter{TempleOS and its Creator}

The objective of this chapter is to introduce Terry A. Davis
and his creation, TempleOS.
It will begin with an explanation of what TempleOS \textit{is}
and what it \textit{does}.
Then, a brief overview will be given of how it is typically installed
and, finally, an effort will be made to convey
the magnitude of the project in order to contextualise
the interest it generated in technical communities.
Following this, Davis will be introduced.
With reference to various primary sources,
a summary will be given of Davis' biography and
an attempt will be made to depict a broad image of his character,
particularly as he presented it towards the end of his life.
The chapter will end with a brief explanation of the initial reception of
the TempleOS project, and its development to the present day.

\section*{The operating system}

Terry A. Davis first began work developing operating systems in 1990
as an employee of Ticketmaster, and started a project he called
``Terry's Protected Mode OS'' in 1993 which was renamed over the years
and eventually released with the name ``TempleOS'' in 2013
\parencite{History}.
Operating systems, like Microsoft's Windows or Apple's macOS,
provide an interface between physical hardware like a laptop or smartphone,
and applications like Microsoft Office or Google Chrome.
They do not typically influence what you \textit{do}
on your computer but, rather, \textit{how} you do those things.
For example, Microsoft's Word application can be installed on both
Windows and macOS, differing primarily in details like keyboard shortcuts.
TempleOS, on the other hand, lacks the capability to perform functions
required by many users in day-to-day use,
like internet browsing and sending emails.
Davis did not develop TempleOS to be a replacement for other operating systems
like Windows, however, but to be installed alongside them.
On the \textit{Welcome to TempleOS} page of TempleOS.org
as it appeared in 2016\footnotemark, Davis \parencite*{Welcome} wrote that
`since this OS is used in addition to Windows or Linux,
[...] failure is an option -- just use Windows if you can't do something.
I cherry-pick what it will and won't do to make it maximally beautiful'.
Moreover, on the \textit{Frequently Asked Questions} page, he explained that
although `TempleOS will work as the only operating system on your computer',
`it has no networking' and 'in your off hours,
you will use your other operating system' \parencite{FAQ}.
In other words, TempleOS is a closed system without access to the internet,
offering only the applications developed and included by Davis:
various games, an application for composing music,
and an oracle that can be accessed from anywhere in the operating system
at the press of a keyboard shortcut which presents the user with randomly
chosen words from a dictionary or passages from the Bible.

\footnotetext{
  A significant number of pages were removed from TempleOS.org over the course
  of 2017 --- a process which resulted in the simple
  single-page website present at the time of writing.
  Although an old version of the website is currently accessible
  at \texttt{https://templeos.holyc.xyz}, a snapshot of Davis' website as it
  appeared on June 1\textsuperscript{st}, 2016
  --- accessed via the Internet Archive's Wayback Machine
  (\texttt{https://archive.org/web}) ---
  will be referenced throughout this dissertation.
}

Indeed, the use of TempleOS can be more helpfully understood in
analogy to the use of applications in general, rather than
in comparison to other more familiar operating systems.
As quoted above, Davis suggested that users could install TempleOS
alongside their existing operating system, but it would more often
be installed in a ``virtual machine'' for the sake of convenience.
The former would necessitate the user to reboot their computer
to access a menu through which they could choose which operating system to
load.
The latter --- employed by Davis as evidenced by many of the
videos he published documenting the system --- involves the use of a special
application within one's existing operating system to create a digital container
in which a second operating system can be installed.
In this way TempleOS can be opened, closed, or minimized in a manner
equivalent to other applications.

Reading its rather ecclectic featurelist, one might wonder for what purpose
Davis created the system, and the answer vaires depending on where you look.
On its introduction page, it describes itself as
an `operating system for recreational programming' most beneficial for
`professionals doing hobby projects', `teenagers doing projects',
and `non-professional, older-persons projects'
\parencite{Welcome}.
In its charter, however, it describes itself as `God's official temple'
and proclaims that
`just like Solomon’s temple, [it] is a community focal point
where offerings are made and God’s oracle is consulted'
\parencite{Charter}.
Indeed, in addition to the OS offering
a malleable environment for recreational programming,
it contains games, Moses comics, and hymns authored by Davis.

\section*{Davis' life}
The ``cult following'' of TempleOS has perhaps
more to do with Davis himself than his creation, however.
The principal source of bibliographical information about him is
an article titled \textit{God’s Lonely Programmer}
in VICE Magazine’s \textit{Motherboard} publication.
In it, Jesse Hicks \parencite*{Hicks14}---drawing
upon `two months of emails and phone conversations'---first
introduces the OS,
but spends the majority of the article de-mystifying the background of Davis.
According to the article,
Davis was born in 1969 in Wisconsin and grew up Catholic.
He started using computers in elementary school
and went on to complete bachelor's and master's degrees
in electrical engineering at Arizona State University.
In his blog, Davis \parencite*{Atheist} wrote that
between 1990 and 1996 he was `as atheist as they come',
and is quoted in Hicks' article saying
`I thought the brain was a computer [...] so I had no need for a soul'.
He goes on to say, however, that he is differentiated from other atheists
because `God has talked to me,
so I'm basically like an atheist who God has talked to'.
Davis describes starting to
`[see] people following me around in suits and stuff' in mid-March 1996
and later ends up in a mental hospital for two weeks
following a run-in with the police.
His mental state stabilises later that year
and he moves back in with his parents,
proceeding to experience manic episodes every six months for the next six years.
The article reports that he hadn't been to the hospital since then, however,
with Davis claiming
`for those first few years, I was genuinely pretty crazy in a way.
Now I'm not. I'm crazy in a different way maybe',
and that he's `learned not to freak out'.
For biographical information beyond this date,
the unofficial ``Timeline of TempleOS'' written by Issa Rice
\parencite*{Rice18} can be consulted.
The timeline 39 entries of bibliographical information
about Davis' ``personal'' life, and a near-equal 40 entries about
the online activities of Davis from 2001 until 2017.
These online activities included his creation---and
often the subsequent suspension---of accounts on various platforms,
and his notable forum posts, video uploads, and live streams.
After living homeless for approximately one year following
his eviction from his family home by his parents,
Davis was hit by a train on the 11th of August 2018 and died \parencite{Cecil18}.


\section*{Davis' online following}
Davis' following can be found across platforms like YouTube and Reddit,
and engage in discussion (serious and humorous) about
technical aspects of the OS, and Davis’ personal life.

\end{document}
