\documentclass[Draft.tex]{subfiles}
\begin{document}
\chapter{TempleOS and its Creator}

The objective of this chapter is to introduce Terry A. Davis
and his creation, the Temple Operating System.
It will begin with an explanation of what the Temple Operating System
\textit{is} and what it can \textit{do}.
Following this, an overview of Davis' biography will be presented
with particular reference to the online article \textit{God's Lonely Programmer}.

\section*{The operating system}

Terry A. Davis first began work developing operating systems in 1990
as an employee of Ticketmaster, and started a project he called
``Terry's Protected Mode OS'' in 1993 which was renamed over the years
and eventually released with the name ``TempleOS'' in 2013
\parencite{History}.
Operating systems, like Microsoft's Windows or Apple's macOS,
provide an interface between physical hardware like a laptop or smartphone,
and applications like Microsoft Office or Google Chrome.
They do not typically influence what you \textit{do}
on your computer but, rather, \textit{how} you do those things.
For example, Microsoft's Word application can be installed on both
Windows and macOS, differing primarily in details like keyboard shortcuts.
The Temple Operating System, on the other hand,
lacks the capability to perform functions required by many users
in day-to-day use, like internet browsing and sending emails.
Davis did not develop the Temple Operating System to be a
replacement for other operating systems like Windows, however,
but to be installed alongside them.
On the ``Welcome to TempleOS'' page of TempleOS.org
as it appeared in 2016\footnotemark, Davis \parencite*{Welcome} wrote that
`since this OS is used in addition to Windows or Linux,
[...] failure is an option -- just use Windows if you can't do something.
I cherry-pick what it will and won't do to make it maximally beautiful'.
Moreover, on the \textit{Frequently Asked Questions} page, he explained that
although `TempleOS will work as the only operating system on your computer',
`it has no networking' and 'in your off hours,
you will use your other operating system' \parencite{FAQ}.
In other words, the Temple Operating System is a closed system
without access to the internet, offering only the applications
developed and included by Davis: various games,
an application for composing music, and an oracle that can be accessed
from anywhere in the operating system at the press of a keyboard shortcut
and presents the user with randomly
chosen words from a dictionary or passages from the Bible.

\footnotetext{
  A significant number of pages were removed from TempleOS.org over the course
  of 2017 --- a process which resulted in the simple
  single-page website present at the time of writing.
  Although an old version of the website is currently accessible
  at \texttt{https://templeos.holyc.xyz}, a snapshot of Davis' website as it
  appeared on June 1\textsuperscript{st}, 2016
  --- accessed via the Internet Archive's Wayback Machine
  (\texttt{https://archive.org/web}) ---
  will be referenced throughout this dissertation.
}

Indeed, the use of TempleOS can be more helpfully understood in
analogy to the use of applications in general, rather than
in comparison to other more familiar operating systems.
As quoted above, Davis suggested that users could install TempleOS
alongside their existing operating system, but it would more often
be installed in a ``virtual machine'' for the sake of convenience.
The former would necessitate the user to reboot their computer
to access a menu through which they could choose which operating system to
load.
The latter --- employed by Davis as evidenced by many of the
videos he published documenting the system --- involves the use of a special
application within one's existing operating system to create a digital container
in which a second operating system can be installed.
In this way TempleOS can be opened, closed, or minimized in a manner
equivalent to other applications.

The value of the Temple Operating System \textit{as an operating system},
then, is not in the programs it includes --- as they could have
been packaged as an application to be installed in Windows or macOS ---
but in the malleable virtual environment the system offers.
The religious mission underwriting Davis' development of the
Temple Operating System will be explained in the following chapter,
but it would be remiss to overlook its less ambitious goal of providing
a tool for `recreational programming' targeted towards
`professionals doing hobby projects', `teenagers doing projects',
and `non-professional, older-persons projects' \parencite{Welcome}.
The Temple Operating System \textit{as an operating system} presents a
`simple machine where programming [is] the goal, not just a means
to an end' and features `a low line count [...]
so it is easy to learn the whole thing' \parencite{Charter}.
The value of the Temple Operating system for such a use-case is
usefully illustrated in a blog post written by Richard Mitton,
`a freelancing British software engineer', titled
``A Constructive Look at TempleOS''.
In it, he references a video in which Davis `shows how to build a small
graphical application from scratch' using a `tiny snippet of code'
and proceeds to explain that the process to achieve the same result
in Microsoft's Windows would be significantly more
complicated, requiring many more lines of code \parencite{CodersNotes}.
Accordingly, `TempleOS is somewhat of a legend in the
operating system community' \parencite{CodersNotes}, not only as
`an educational tool for programming experiments' but also
`a testament to the dedication and passion of one man
displaying his technological prowess' \parencite{TechRepublic}.
In his obituary article in \textit{The Dalles (Oregon) Chronicle},
Davis' single-handed development of the Temple Operating System was likened to
`building a skyscraper by yourself' \parencite{Cecil18}.

\section*{Davis' life}
The ``cult following'' of TempleOS has perhaps
more to do with Davis himself than his creation, however.
The principal source of ``offline'' biographical information about him is
an article titled \textit{God’s Lonely Programmer}
in VICE Magazine’s \textit{Motherboard} publication.
In it, Jesse Hicks \parencite*{Hicks14}---drawing
upon `two months of emails and phone conversations'---first
introduces the OS,
but spends the majority of the article de-mystifying the background of Davis.
According to the article,
Davis was born in 1969 in Wisconsin and grew up Catholic.
He started using computers in elementary school
and went on to complete bachelor's and master's degrees
in electrical engineering at Arizona State University.
On the blog section of his website, Davis \parencite*{Atheist} wrote that
between 1990 and 1996 he was `as atheist as they come',
and is quoted in Hicks' article saying
`I thought the brain was a computer [...] so I had no need for a soul'.
He goes on to say, however, that he is differentiated from other atheists
because `God has talked to me,
so I'm basically like an atheist who God has talked to'.
Davis describes starting to
`[see] people following me around in suits and stuff' in mid-March 1996
and later ends up in a mental hospital for two weeks
following a run-in with the police.
His mental state stabilises later that year
and he moves back in with his parents,
proceeding to experience manic episodes every six months for the next six years.
The article reports that he hadn't been to the hospital since then, however,
with Davis claiming
`for those first few years, I was genuinely pretty crazy in a way.
Now I'm not. I'm crazy in a different way maybe',
and that although he had `since been declared schizophrenic' he
`[shrugged] off the diagnosis', saying he had `learned not to freak out'.

The unofficial ``Timeline of TempleOS'' written by Issa Rice
\parencite*{Rice18} offers a useful overview of the known
biographical data about Davis, listing 39 entries of biographical information
about Davis' ``personal'' life alongside a near-equal 40 entries about
the online activities of Davis from 2001 until 2017.
These online activities included his creation --- and
often the subsequent suspension --- of accounts on various platforms,
and his notable forum posts, video uploads, and live streams.
According to this timeline, Davis' first live stream ---
a medium through which a ``streamer'' broadcasts video live on the internet
and hosts a chatroom for viewer interaction --- took place on March 16, 2016.
That first live stream lasted for an hour and a half and featured Davis
working on the development of the Temple Operating System,
using it to compose a hymn tune, reading various popular media websites,
updating the blog page on his website, and asking God questions
like `did dinosaurs tangle their necks?', divining answers through the
random words produced by his oracle program
\parencite[59:05]{FirstLiveStream}.
Most of his live streams included similar activities to his first,
although he would also lapse into extended monologues about
conspiracy theories regarding the CIA and contemplation of his place in
reality, comparing himself to the protagonist in \textit{The Truman Show}
--- a film about a businessman whose entire life has been
orchestrated around him as part of a reality TV show ---
in one notable video titled \textit{Reality} \parencite{Reality}.
Davis' monologues became increasingly existential and incoherent following
his ejection from the family home by his parents following a fight
with his father in September 2017 \parencite{Homeless}.
Davis lived in a van for approximately one year following this incident
until he was hit by a train on August 11, 2018, and died \parencite{Cecil18}.

\end{document}
