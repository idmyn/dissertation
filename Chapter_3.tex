\documentclass[Draft.tex]{subfiles}

\begin{document}
\chapter{An Analysis of the Intention for and Reception of TempleOS}

The objectives of this chapter are to first examine
the intentions of Terry A. Davis in the creation of TempleOS and, second,
to analyse its reception in various online communities.
I hope to show that although Davis was serious in his invention of TempleOS,
its religious content and motivations were dismissed
in the communities that expressed technical interest and, finally,
that although communities formed around Davis make frequent reference
to its religious content, they do so in jest.

\section*{Terry A. Davis, the prophet}

\begin{displayquote}
  I am Roman Catholic, and the religion for my kingdom
  would be Roman Catholic, but the church is dead [...].
  Right now the computer industry is sick like Herod,
  the Roman Church is sick like Herod,
  the world is sick like Herod, and I'm going to cure it.
\end{displayquote}

This was a proclamation made by Terry A. Davis \parencite*{NewReligionVid}
in a video titled \textit{I'm Starting a New Religion}
uploaded a month before his death.
In it, Davis stands outside a church and speaks informally to the camera,
weaving between the topics of his religious mission
and the beauty of the church beside him.
%Davis' aspiration to reform the computer industry
%will be briefly considered first, before a deeper consideration
%of his religious mission.

\section*{Davis' Christian mission}
While TempleOS' charter offers a lucid summary of its technical aspirations,
a clear formulation of its religious mission is harder to find.
The most complete written statement of its religious mission can be found
following the links ``About Terry Davis'' and ``New Religion'' on the
TempleOS.org as it appeared in mid-2016.
The page begins with the heading ``Christianity, a Charity or a Church?''
under which Davis \parencite*{NewReligionSite} quotes John 2:14-17, in which
Jesus drives merchants out of the temple in Jerusalem,
and writes that `today, Christianity is a secular humanist social club
that does charity.  A church should primarily love God and do prayers'.
Following this, alongside two more sections of bible passages in which
the words ``oracle'' and ``enquire'' are highlighted,
Davis proclaims: `I built God's temple.  It is a place for prayer, offering and
enquiring of God using a divine random oracle'.
His vision for the future can be found under the heading
``The Counter-Renaissance'':
`I have a dream that obsession with God in the United States
will return to the level it was in Europe in 1200 A.D.
Europeans built cathedrals and had monastaries
that wrote beautifully decorated books.
I dream people will strive to make God's temple beautiful, glorious
and as perfect as possible.  It will be adored.
People will do offerings in God's temple and God will talk'.

Davis appeared to see his role as more than a software developer
assembling a tool to promote and empower Christian practice, however.
At the bottom of the ``New Religion'' page,
under the heading ``24/7 Continual offerings'',
he wrote `I, Terry A. Davis, am High Priest of God's Temple.
I maintain the 100,000 core lines of code and I do 24/7 offerings'.
He also presents himself a `the Father of the Counter-Renaissance',
calling for an acceptance of this world as ``perfectly just''
and a rejection of rationality.
For example, he calls for the replacement of charity
--- a ``rational'' action ---
with prayer --- an ``irrational'' one ---
and suggests that while `a Renaissance man
attempts to use philosophy to prove God exists',
`a Counter-Renaissance man knows that belief in God comes,
not [from] philosophy, but from the occult'
\parencite{CounterRenaissance}.
It seems, then, that Davis considered himself to be
a new religious authority with ambition to reform Christianity.
He expressed discomfort at the thought of being identified as a prophet,
however, in a video titled ``Terry's Sermon on the Mount'':
`God talked to me, right?
So obviously I asked myself: ``am I like the prophets in the bible?''
You know? I kinda disrepect the mormons.
The scientology people, I don't like them.
So I hesitate to step up and say ``look I'm a prophet and I'm on their level'',
ok? I hesitate to do that, I know what it looks like.
Everyone says ``ohh, who are you? Who are you? You're not a prophet'',
you know, ok, I understand that' \parencite{SermonMount}.

Davis' reported contact with God is centered upon his use of
a ``random oracle'', the process of which he explained in a video titled
``TempleOS: NIST Random Beacon'' \parencite{NIST}.
Simply put, it involves using a randomly generated number to select
words from a dictionary or passages from the Bible.

Davis' role as ``high priest'' of TempleOS, in his mind, granted him authority
over the computer industry.
On a page titled ``Demands'' Davis claimed
`I have divine authority to command any company in the computer industry
to do anything that I deem necessary
to make God's temple more beautiful, glorious and perfect',
and drew an analogy of this role with that of a building inspector
or ``enforcer of the Disability Act'' \parencite{Demands}.
For example, Davis demands that the Windows and Linux operating systems
develop support for TempleOS' ``RedSea'' file system so that he can
`get rid of 2,000 lines of redundant, blemished code [...]
God's temple must be perfect.  Redundant code for multiple file systems
is imperfect.  For this operating system, we want low line count'.

\section*{Reception}
This section will consider the reception of Terry Davis in various groups
ranging from unintentional (youtube) to intentional (/r/TempleOS\_Official).

The most-viewed video included in search results on YouTube
for the queries ``Terry Davis'' and ``TempleOS'' is
\textit{TempleOS | Down the Rabbit Hole} (henceforth \textit{DRH})
by Fredrik Knudsen with 1.6 million views
and a running time of one hour and 25 minutes.
The next highest viewcount is \textit{Internet Insanity: Terry A. Davis}
(henceforth \textit{II}) with 761 thousand views and a running time of 17 minutes.
Both of these videos were uploaded as episodes of ongoing series
--- ``Down the Rabbit Hole'' and ``Internet Insanity'', respectively ---
on YouTube channels with hundreds of thousands of subscribers.
These are videos, then, that were watched by a large audience,
a significant proportion of which had likely never heard of Davis
or TempleOS before, as can be infered
by the significantly lower view-counts (less than 200 thousand)
on other videos appearing in the same search.
Although the content of the videos themselves
will not be discussed in great detail here,
it is worth noting their difference in tone for context.
\textit{II} was uploaded to YouTube in mid-2017 and adopts a damning stance
towards Davis' mental illness, introducing him as
`a man who had a psychotic break.
A talented man, but [one] who encountered mental illness so severe
that any legacy he has will be overshadowed
by the sideshow that he has become' \parencite*{Metokur17}.
Contrastingly, Knudsen's \parencite*{Knudsen18} video
takes a more balanced approach, introducing TempleOS as,
`depending on who is consulted, [...] the outdated product of a deranged mind,
the work of a misunderstood genius, or some complicated combination of the two'.



%\begin{displayquote}
%  I didn't start the operating system as a work for God,
%  but He directed my path along the way and kept saying it was His temple.
%  Still I hesistated and kept it secular until, finally,
%  Microsoft went nuclear with SecureBoot and UEFI.
%  Then, I went nuclear and named it "TempleOS".
%  I will command them on orders from God to UNDO THAT STUFF!
%  [TempleOS History templeos.holyc.xyz]
%\end{displayquote}


\end{document}
