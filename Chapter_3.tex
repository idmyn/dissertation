\documentclass[Draft.tex]{subfiles}

\begin{document}
\chapter{An Analysis of the Intention for and Reception of TempleOS}

The objectives of this chapter are to first examine
the intentions of Terry A. Davis in the creation of TempleOS and, second,
to analyse its reception in various online communities.
I hope to show that although Davis was serious in his invention of TempleOS,
its religious content and motivations were dismissed
in the communities that expressed technical interest and, finally,
that although communities formed around Davis make frequent reference
to its religious content, they do so in jest.

\section*{Terry A. Davis, the prophet}

\begin{displayquote}
  I am Roman Catholic, and the religion for my kingdom
  would be Roman Catholic, but the church is dead [...].
  Right now the computer industry is sick like Herod,
  the Roman Church is sick like Herod,
  the world is sick like Herod, and I'm going to cure it.
\end{displayquote}

This was a proclamation made by Terry A. Davis \parencite*{NewReligionVid}
in a video titled \textit{I'm Starting a New Religion}
uploaded a month before his death.
In it, Davis stands outside a church and speaks informally to the camera,
weaving between the topics of his religious mission
and the beauty of the church beside him.
There is a definite sense in which Davis' mission was twofold;
he wanted to spread the word of God, and to reform the computer industry.

\section*{Davis' Christian mission}
While TempleOS' charter offers a lucid summary of its technical aspirations,
a clear formulation of Davis' personal mission is harder to find.
The most complete written statement of Davis' religious mission can be found
following the links ``About Terry Davis'' and ``New Religion'' on the
TempleOS.org as it appeared in mid-2016.
The page begins with the heading ``Christianity, a Charity or a Church?''
under which Davis \parencite*{NewReligionSite} quotes John 2:14-17, in which
Jesus drives merchants out of the temple in Jerusalem,
and writes that `today, Christianity is a secular humanist social club
that does charity.  A church should primarily love God and do prayers'.
Following this, alongside two more sections of bible passages in which
the words ``oracle'' and ``enquire'' are highlighted,
Davis proclaims `I built God's temple.  It is a place for prayer, offering and
enquiring of God using a divine random oracle'.
From this, one can derive that Davis developed TempleOS as a tool for
Christian prayer --- something he felt was insufficinetly practiced in church.
He communicates his vision for the future under the heading
``The Counter-Renaissance'':
`I have a dream that obsession with God in the United States
will return to the level it was in Europe in 1200 A.D.
Europeans built cathedrals and had monastaries
that wrote beautifully decorated books.
I dream people will strive to make God's temple beautiful, glorious
and as perfect as possible.  It will be adored.
People will do offerings in God's temple and God will talk'.
Davis includes a link at the end of this section to a section that
explicates his ``counter-renaissance'' position further.
In short, Davis calls for a rejection of rationality.
He suggests charity --- a ``rational'' action ---
should be replaced by prayer --- an ``irrational'' one ---
and suggests that while `a Renaissance man
attempts to use philosophy to prove God exists',
`a Counter-Renaissance man knows that belief in God comes,
not [from] philosophy, but from the occult'.


%\begin{displayquote}
%  I didn't start the operating system as a work for God,
%  but He directed my path along the way and kept saying it was His temple.
%  Still I hesistated and kept it secular until, finally,
%  Microsoft went nuclear with SecureBoot and UEFI.
%  Then, I went nuclear and named it "TempleOS".
%  I will command them on orders from God to UNDO THAT STUFF!
%  [TempleOS History templeos.holyc.xyz]
%\end{displayquote}


\end{document}
