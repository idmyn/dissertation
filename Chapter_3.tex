\documentclass[Draft.tex]{subfiles}

\begin{document}
\chapter{An Analysis of the Intention Behind and Reception of TempleOS}

The objectives of this chapter are to first examine
the intentions of Terry A. Davis in the creation of TempleOS and, second,
to analyse its reception in various online communities.
I hope to show that although Davis was serious in his invention of TempleOS,
its religious content and motivations were dismissed
in the communities that expressed technical interest and, finally,
that although communities formed around Davis make frequent reference
to its religious content, they do so in jest.


\section*{Terry A. Davis, high priest of God's official temple}

\begin{displayquote}
  I am Roman Catholic, and the religion for my kingdom
  would be Roman Catholic, but the church is dead [...].
  Right now the computer industry is sick like Herod,
  the Roman Church is sick like Herod,
  the world is sick like Herod, and I'm going to cure it.
	(Terry A. Davis, \textit{I'm Starting a New Religion} [Internet Archive, 2018])
\end{displayquote}

TempleOS is introduced at the beginning of its charter as
`God's official temple. [...] [A] community focal point
where offerings are made and God's oracle is consulted' \parencite{Charter}.
The remainder of the document, however, is dedicated to technical details,
such as the `limit of 100,000 lines of code for all time'
and `just one 8x8 fixed-width font'.
Further explanation of its religious purpose is located somewhat sporadically
on other pages on the TempleOS website, and in videos uploaded by Davis.
In \textit{The 64-Bit Operating System}, for example,
Davis \parencite*{64-Bit} introduced TempleOS
with reference to its oracle feature --- which uses a high-speed
stopwatch to select random words and bible passages --- suggesting that its
`main purpose is for doing offerings of hymns and art and poems and stuff
and then getting a response from God in the oracle'.

Davis' broader religious mission appears to have been to reform
Christianity by enabling easy communication between lay-people and God.
On a page titled \textit{New Religion}, and under the heading
``Christianity, a Charity or a Church?'', Davis \parencite*{NewReligion}
wrote that `today, Christianity is a secular humanist social club
that does charity.  A church should primarily love God and do prayers'.
Further down, under the heading ``The Counter-Renaissance'',
Davis described his `dream that obsession with God in the United States
will return to the level it was in Europe in 1200 A.D.
Europeans built cathedrals and had monastaries
that wrote beautifully decorated books. [...]
people will strive to make God's temple beautiful, glorious
and as perfect as possible.  It will be adored.
People will do offerings in God's temple and God will talk'.
On another page titled \textit{The Purpose of Life},
Davis \parencite*{PurposeLife} commanded `You don't know God. [...]
You must talk with God to know Him. [...] Seek the Lord by taking initiative',
and proceeded to explain his rationale as follows:
\begin{displayquote}
	There's something obviously different about people in the Bible
	compared to people today -- God talked!
	Also, the people in the Bible were obsessed with doing offerings all the time.
	It is required that you do offerings before God will talk.
	Did the people in the Bible hear voices?  Maybe.
	More likely, they used occult techniques such as an oracle.
\end{displayquote}
It appears, then, that Davis wanted to, in some sense, reform Christianity
with a renewed focus on communication with God
as it is presented in the Old Testament ---
by giving offerings and consulting an oracle.

Furthermore, Davis \parencite*{Demands} understood himself to have a role
more significant than that of an ordinary software developer,
introducing himself on his \textit{Demands} page as
`high priest of God's official temple, TempleOS' with
`divine authority to command any company in the computer industry
to do anything that [he] deem necessary to make God's temple
more beautiful, glorious and perfect'.
His obligations in this role were twofold: he was
`in charge of the core 100,000 lines of TempleOS code'
and he did `continual offerings to keep God entertained'.


\section*{Reception}
This section will consider the reception of Terry Davis in various groups
ranging from unintentional (youtube) to intentional (r/TempleOS\_Official).

The most-viewed video included in search results on YouTube
for the queries ``Terry Davis'' and ``TempleOS'' is
\textit{TempleOS | Down the Rabbit Hole} (henceforth \textit{DRH})
by Fredrik Knudsen with 1.6 million views
and a running time of one hour and 25 minutes.
The next highest viewcount is \textit{Internet Insanity: Terry A. Davis}
(henceforth \textit{Insanity}) with 761 thousand views and a running time of 17 minutes.
Both of these videos were uploaded as episodes of ongoing series
--- ``Down the Rabbit Hole'' and ``Internet Insanity'', respectively ---
on YouTube channels with hundreds of thousands of subscribers.
These are videos, then, that were watched by a large audience,
a significant proportion of which had likely never heard of Davis
or TempleOS before, as can be infered
by the significantly lower view-counts (less than 200 thousand)
on other videos appearing in the same search.
Although the content of the videos themselves
will not be discussed in great detail here,
it is worth noting their difference in tone for context.
\textit{Insanity} was uploaded to YouTube in mid-2017 and adopts a damning stance
towards Davis' mental illness, introducing him as
`a man who had a psychotic break.
A talented man, but [one] who encountered mental illness so severe
that any legacy he has will be overshadowed
by the sideshow that he has become' \parencite*{Metokur17}.
Contrastingly, Knudsen's \parencite*{Knudsen18} video
was uploaded a few months after Davis' death and
takes a more balanced approach, introducing TempleOS as,
`depending on who is consulted, [...] the outdated product of a deranged mind,
the work of a misunderstood genius, or some complicated combination of the two'.

Commentor Hey Trey (2019) aptly summarises
the narrative arc of Knudsen's portrayal of Davis:
\begin{displayquote}
	i started the video really disliking the guy,
	thinking to myself what a smug pos [piece of shit].
	transitiones [\textit{sic}] into, this guy is crazy and racist, and comedy gold.
	Then, as the video went on i went from laughing, to absolute total pity
	and a feeling of total sadness that he was very obviously
	becoming very unhealthy and mentally unstable.
	and it all culminated with absolute depression and confusion as to
	how someone so smart could go down such a crazy winding path
	that led to their suicide.
\end{displayquote}
Perhaps resultingly, the comments on this video, when sorted by ``top comments''
to prioritise those given the most ``thumbs up'',
are overwhelmingly sympathetic towards Davis.

The comments on \textit{Insanity} fall broadly into four categories according
to their character.
There are sympathetic comments --- many of which commemorating his death ---
humourous comments, comments admiring Davis' talent, and comments that
gesture towards a power or truth in Davis' purported worldview.

Sympathetic comments can also be found on \textit{Insanity} --- particularly
since Davis' death --- but the less sympathetic tone of the video itself
results in more varied comments.
Andrew Devito (2018) provides an example of a sympathetic comment
on \textit{Insanity}, writing `Rest in Peace Terry, you'll be missed.
For once, I actually felt pity for someone on the internet.
It's terrible what a mental illness really can do. Sleep well prince'.
Many of the humour-oriented comments reference computing ---
Mayor of Gaming (2018) writes `As a programmer, I'm positive that
a few of my colleagues are 1 compile error away from becoming this guy'.
Of course, many comments bridge more than one theme.
There are some comments highlighting Davis' technical ability without
explicitly expressing sympathy, but the majority appear to do so.
For example, Teamugi (2018) writes
`Schizophrenia aside, this guy has talent.
Building an OS from scratch by yourself [...] is extremely impressive',
but Jeferson Oliveria (2018) writes
`His genius transcended his mental illness [...] A true loss.
RIP, man.  Your legacy will live on'.
Similarly, many of the comments suggesting Davis was ``on to something''
with his worldview are expressed with humerous orientation.
Heinrich Cornelius Agrippa (2018) writes `With this character's death,
the thread of prophecy is severed.
Restore a saved game to restore the weave of fate,
or persist in the doomed world you have created' --- referencing the
message received by players in Morrowind (CITE?) if they kill an
``essential character'' in a quest.

Darkwing Dumpling (2018) writes
`Terry is the one who coded the matrix[.] We need him'.
\textit{Computer/gaming culture references.}

Others appear more sincere.
\foreignlanguage{russian}{братишка} (2018) writes `if you watch his ramblings long enough,
they actually start to make sense. im serious. try it'


The comments left by viewers on YouTube can be usefully contrasted
with threads found on the TempleOS\_Official subreddit.
Subreddits are community pages formed around topics of interest on
Reddit.com and are denoted by the ``r/'' prefix --- the
TempleOS subreddit, for example, is named r/TempleOS\_Official.
Reddit's homepage features posts from some of the most popular
subreddits by default --- r/books, for example ---
and the ``default subreddit'' /all is a compilation of the most popular
posts from a wide variety of subreddits measured by subtracting their
``downvotes'' from their ``upvotes'',
but posts on the vast majority of the roughly 1.2 million subreddits
are only seen by those who visit them directly.
If user wishes to see posts from a particular subreddit on their
homepage, they can do so by subscribing.
For scale, r/science, a default subreddit to which all new accounts
are subscribed, has roughly 18 million subscribers,
r/soccer 850 thousand, and r/TempleOS\_Official 1,800.
Notably, the highest rated post of all time on r/TempleOS\_Official
is a link to Knudsen's \textit{TempleOS | Down the Rabit Hole} video,
posted by Klyke \parencite*{Klyke18},
and some of the comments on the post highlight the presence of
an intentional community.
BiggRanger (2018) wrote `Seems like a pretty well done video [...].
The guy did his homework on this,
but still has the feel of an outsider looking in'.
Another commentor, GDP10, writes
`I don't really like how at some points this video portrays Terry as a madman.
It's not really fair.  Many of his moments of ``insanity''
are actually quite lucid and he has sound points. [...]
Rest in peace Terry.  I know you can finally talk with God face to face'.
Not only does GDP10 defend Davis' ramblings, but he quotes a comment from
another user on a different post on the subreddit made four months earlier and,
in a second comment made an hour later includes a quote from a comment made
on a post about TempleOS on r/programming five years ago ---
the first emphasising the difficulty of living with a mental illness like
Davis', and the second encouraging others not to dismiss Davis'
technical achievements on account of `his condition'.

\end{document}
