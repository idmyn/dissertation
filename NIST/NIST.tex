\documentclass{article}
\usepackage[authordate,backend=biber]{biblatex-chicago}
\addbibresource{../diss.bib}

\usepackage{setspace}
\doublespacing
\usepackage[activate={true,nocompatibility},final,tracking=true,kerning=true,spacing=true,factor=1100,stretch=10,shrink=10]{microtype}

\usepackage{amssymb}

\author{David Mynors}
\title{Davis and NIST}
\date{2019-01-17}

\begin{document}
\section*{Divination}

Fritz Graf \parencite{Divination11} derives the term `divination'
from the Latin \textit{divinare}, ``to ascertain the divine will'',
and describes it as ``a procedure employed in many religious cultures
to discover by ritual means what is hidden from human knowledge''.
He goes on two explain that divination practices function
either directly, or indirectly.
The first category involves the concerned individual experiencing
special dreams or ecstasy--possibly against their will--and interpreting
those experiences in order to ascertain communication from a deity.
The latter category involves the interpretation of
``a number of arbitrary signs'' as a message from a deity.

\clearpage
\printbibliography
\end{document}
