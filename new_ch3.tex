  \documentclass{article}
  \usepackage[utf8]{inputenc}

  \usepackage[authordate,backend=biber]{biblatex-chicago}
  \addbibresource{diss.bib}
  \DeclareLabeldate{\field{date}\field{eventdate} \field{origdate}\literal{nodate}}

  \usepackage{csquotes}
  \usepackage{endnotes}
  \let\footnote=\endnote

  \usepackage[activate={true,nocompatibility},final,tracking=true,kerning=true,spacing=true,factor=1100,stretch=10,shrink=10,letterspace=50]{microtype}
  \microtypecontext{spacing=nonfrench}
  \usepackage[T1]{fontenc}
  \usepackage[bitstream-charter]{mathdesign}
  \usepackage{inconsolata}

  \usepackage{setspace}
  \setstretch{1.25}
  \usepackage{parskip}
  %\setlength{\parindent}{2em}

  \setlength{\skip\footins}{20pt}

  \widowpenalty10000
  \clubpenalty10000
\begin{document}


Steven Sutcliffe \parencite*{Sutcliffe14},
in \textit{New Age Spirituality: Rethinking Religion},
suggests that there is a `widespread perception that [...]
new age beliefs and practices [...] appear to constitute
a major ``unruly object'' in the study of religion'.
This final chapter hopes to position the following of TempleOS as a
similarly ``unruly object'': a liminal case able to contribute to the
important boundary-work in religious studies.
The process of boundary-work will first be introduced,
alongside an exposition of liminality via Arnold van Gennep and Victor Turner.
Sutcliffe will fit into this somewhere.
Polythetic definitions will be introduced via Cusack and invented religions,
which are also slippery characters who benefit from such a definition.
Some of TempleOS' boundary transgressions will be introduced
(online--offline, sacred--secular, and mechanical--spiritual)
and the mechanical--spiritual tension will be explored
against with the backdrop of Erik Davis' \textit{Techgnosis}
and in constrast to the counter-examples of astrology and tarot applications.

Thomas F. Gieryn \parencite*[781]{Gieryn83}, writing in an article
addressing the demarcation of science from non-science, aptly observes
that although demarcation `is routinely accomplished in practical,
everyday settings' such as school curricula,
philosophers and sociologists of science have long struggled
with the challenge of identifying `unique and essential characteristics'
which distinguish it definitively from non-science.
He goes on to recognise boundary-work as part of a `rhetorical style [...]
sometimes hoping to enlarge the material and symbolic resources of scientists
or to defend professional autonomy' \parencite[782]{Gieryn83}.
In the case of religion, to be classified one way or the other can similarly
be a matter of real importance to organisations and individuals,
and is often negotiated on their behalf.
Talal Asad \parencite*[39]{Asad11} summarises this power
in the past and present thus:
`in the past, colonial  administrations  used definitions of religion
to classify, control, and regulate the practices and identities of subjects.
Today, liberal democracy  is required to pronounce on the legal status
of such definitions and thus to spell out civil immunities and obligations',
and he notes that `academic expertise is often invoked
in the process of arriving at legal decisions about religious matters'.

How is it done? Examples?

This chapter looks specifically at contemporary boundary-trouble
in the definition of religion, and will do so through an investigation
of liminal boundary cases.
Sutcliffe, as mentioned in the introduction to this chapter,
provides a useful overview of the contemporary difficulties through his
identification of the trouble encountered in efforts to situate
``new age data''.
Sutcliffe writes that `the ``world religions'' paradigm is embedded in the
basic fabric of our thinking about religion' and that it causes problems
as a classification system through its authorisation of the reproduction of
`only a limited set of approved religion entities'
\parencite[22]{Sutcliffe14}.
This results in a ``structural disadvantage'' for
`local, small-scale, diffuse, informal, situational, hybrid and syncretic'
religious formations \parencite[25]{Sutcliffe14}.
Furthermore, `The taxonomy is further reinforced by
its recognition and approval on the part of many religious organizations,
which can derive practical advantage in the form of enhanced social capital
and related benefits in so far as they are able to position
their own traditions in ``world'' terms' \parencite[23]{Sutcliffe14}.
The more entrenched the ``world religion'' model becomes,
the more difficult it becomes to `conceptualize a different paradigm
of classification of the data for ``religion''\thinspace'
\parencite[25]{Sutcliffe14}.

Carol Cusack develops a significant theoretical framework encompassing
a particular form of boundary case in \textit{Invented Religions}.
Invented religions in Cusack's \parencite*[1]{Cusack10} definition,
although `they can be seen to be functionally similar,
if not identical, to traditional religions',
openly declare their invented status in a rejection of the
`web of conventions that surround the establishment of new religions'.
Their beliefs might be structured around the content of a novel,
as in the case of the Church of All Worlds (53),
or perhaps they emerged as a product of late-night discussion in
24-hour bowling alleys, in the case of Discordianism (28).
Rather than legitimate the religious status of these formations
with reference to the ``world religions'' model, she references
the `Peter Berger and Thomas Luckmann’s
social-constructionist model of reality-building' (HOW) in order to suggest
that `there is a sense in which every religion has been invented'
\parencite{CusackPodcast}, and employs a polythetic definition of religion (WHAT).
These will be considered in turn.

Berger and Luckmann \parencite*[79]{Berger67} wrote in
\textit{The Social Construction of Reality} that
`Society is a human product. Society is an objective reality.
Man is a social product'.
Berger later applied that dialectic model of society to religion
in \textit{The Social Reality of Religion},
and it is the summary he gives of the theory
in the latter book which will be reproduced here.
The dialectic model of society suggests that
`the two statements, that society is the product of man
[\textit{sic}] and that
man is the product of society, are not contradictory'
\parencite[3]{Berger69}.
Put another way, Berger (3) observes that although
`there can be no social reality apart from man',
`it is within society, and as a result of social processes,
that the individual becomes a person,
that he attains and holds onto an identity,
and that he carries out the various projects that constitute his life'.

This dialectic relationship between humans and society is said to operate
through a three-step process of
`externalization, objectivation, and internalization' (3--4).
Externalization is the moment in which humans create society
through the `ongoing outpouring' of physical and mental activity
into the world (4).
The human origin of externalization's product is obscured
in the moment of objectivation, however,
resulting in its objective appearance (4).
Finally, it is through internalization that the external
reality of society influences humans, resulting in
the socialisation of the individual
`\textit{to be} a designated person
and to \textit{inhabit} a designated world'
and thus providing a `meaningful order, or nomos' (16, 19).
The role played by religion in this world-building process
is that of ``cosmization'', that is,
`the identification of [the] humanly meaningful world [nomos]
with the world as such, [...] reflecting it or being derived from it
in its fundamental structures' (28).
In other words, religion is a process through which humans
come to take society for granted as part of ``the nature of things''
--- the cosmos --- in order to stabilise their socially-constructed reality (25).
In this way, `human order is projected into the totality of being'
and `the entire universe [is conceived of] as being humanly significant' (28).

Religion, understood through this framework as a vehicle for cosmization,
`is, to a large extent, about narrative
and the success of the story' \parencite[4]{Cusack10}.
With this in mind, and in light of Cusack's (4) observation
that the most resonant stories of the twentieth century
fell broadly in the science fiction genre,
one can see how a religion framed by the cosmology of
the \textit{Star Wars} films, for example, might emerge.



\printbibliography
\end{document}
