  \documentclass{article}
  \usepackage[utf8]{inputenc}

  \usepackage[authordate,backend=biber]{biblatex-chicago}
  \addbibresource{diss.bib}
  \DeclareLabeldate{\field{date}\field{eventdate} \field{origdate}\literal{nodate}}

  \usepackage{csquotes}
  \usepackage{endnotes}
  \let\footnote=\endnote

  \usepackage[activate={true,nocompatibility},final,tracking=true,kerning=true,spacing=true,factor=1100,stretch=10,shrink=10,letterspace=50]{microtype}
  \microtypecontext{spacing=nonfrench}
  \usepackage[T1]{fontenc}
  \usepackage[bitstream-charter]{mathdesign}
  \usepackage{inconsolata}

  \usepackage{setspace}
  \setstretch{1.25}
  \usepackage{parskip}
  %\setlength{\parindent}{2em}

  \setlength{\skip\footins}{20pt}

  \widowpenalty10000
  \clubpenalty10000
\begin{document}


Steven Sutcliffe \parencite*{Sutcliffe14},
in \textit{New Age Spirituality: Rethinking Religion},
suggests that there is a `widespread perception that [...]
new age beliefs and practices [...] appear to constitute
a major ``unruly object'' in the study of religion'.
This final chapter hopes to position the following of TempleOS as a
similarly ``unruly object'': a liminal case able to contribute to the
important boundary-work in religious studies.
The process of boundary-work will first be introduced,
alongside an exposition of liminality via Arnold van Gennep and Victor Turner.
Sutcliffe will fit into this somewhere.
Polythetic definitions will be introduced via Cusack and invented religions,
which are also slippery characters who benefit from such a definition.
Some of TempleOS' boundary transgressions will be introduced
(online--offline, sacred--secular, and mechanical--spiritual)
and the mechanical--spiritual tension will be explored
against with the backdrop of Erik Davis' \textit{Techgnosis}.

Thomas F. Gieryn \parencite*[781]{Gieryn83}, writing in an article
addressing the demarcation of science from non-science, aptly observes
that although demarcation `is routinely accomplished in practical,
everyday settings' such as school curricula,
philosophers and sociologists of science have long struggled
with the challenge of identifying its `unique and essential characteristics'.
He goes on to recognise boundary-work as part of a `rhetorical style [...]
sometimes hoping to enlarge the material and symbolic resources of scientists
or to defend professional autonomy' \parencite[782]{Gieryn83}.
In the case of religion, to be classified one way or the other can similarly
be a matter of real importance to organisations and individuals,
and is often negotiated on their behalf.
Talal Asad \parencite*[39]{Asad11} summarises this power
in the past and present thus:
`In the past, colonial  administrations  used definitions of religion
to classify, control, and regulate the practices and identities of subjects.
Today, liberal democracy  is required to pronounce on the legal status
of such definitions and thus to spell out civil immunities and obligations',
and he notes that `academic expertise is often invoked
in the process of arriving at legal decisions about religious matters'.

How is it done?

This chapter looks specifically at contemporary boundary-trouble
in the definition of religion, and will do so through an investigation
of liminal boundary cases.
Sutcliffe, as mentioned in the introduction to this chapter,
provides a useful overview of the contemporary difficulties through his
identification of the trouble encountered in efforts to situate
``new age data''.
Sutcliffe writes that `the ``world religions'' paradigm is embedded in the
basic fabric of our thinking about religion' and that it causes problems,
as a classification system through its authorisation of the reproduction of
`only a limited set of approved religion entities'
\parencite[22]{Sutcliffe14}.

\printbibliography
\end{document}
