\documentclass[Draft.tex]{subfiles}
\begin{document}
\chapter{Boundary Work}

This final chapter hopes to position TempleOS as a
valuable boundary case for the definition of religion.
First, the process of boundary work will be introduced,
followed by a brief exposition of a contemporary case of boundary trouble
in the study of religion --- new age spiritualities.
The definitional work performed by Carol Cusack in the context of
``invented religions'' will then be presented, including a brief summary of
Peter Berger and Thomas Luckmann's social-constructionist model of
world-building and the difference between essentialist and polythetic
definitions of religion.
Finally, with extensive reference to Erik Davis' \textit{TechGnosis},
TempleOS' value as a boundary case with be elucidated through
an explanation of its transgression of the intuitive mechanical--spiritual
binary.

Thomas F. Gieryn \parencite*[781]{Gieryn83}, writing in an article
addressing the demarcation of science from non-science, aptly observes
that although demarcation `is routinely accomplished in practical,
everyday settings' such as school curricula,
philosophers and sociologists of science have long struggled
with the challenge of identifying `unique and essential characteristics'
which distinguish it definitively from non-science.
He goes on to recognise boundary work as part of a `rhetorical style [...]
sometimes hoping to enlarge the material and symbolic resources of scientists
or to defend professional autonomy' \parencite[782]{Gieryn83}.
In the case of religion, to be classified one way or the other can similarly
be a matter of real importance to organisations and individuals,
and is often negotiated on their behalf.
Talal Asad \parencite*[39]{Asad11} summarises this power
in the past and present thus:
`in the past, colonial  administrations  used definitions of religion
to classify, control, and regulate the practices and identities of subjects.
Today, liberal democracy  is required to pronounce on the legal status
of such definitions and thus to spell out civil immunities and obligations',
and he notes that `academic expertise is often invoked
in the process of arriving at legal decisions about religious matters'.

A contemporary example of this can be found in Steven Sutcliffe's
\textit{New Age Spirituality: Rethinking Religion}.
Sutcliffe writes that `the ``world religions'' paradigm is embedded in the
basic fabric of our thinking about religion' and that --- through its
authorisation of the reproduction of `only a limited set
of approved religion entities' --- it causes ``structural disadvantage'' for
`local, small-scale, diffuse, informal, situational, hybrid and syncretic'
religious formations \parencite[22, 25]{Sutcliffe14}.
Moreover, `The taxonomy is further reinforced by
its recognition and approval on the part of many religious organizations,
which can derive practical advantage in the form of enhanced social capital
and related benefits in so far as they are able to position
their own traditions in ``world'' terms' \parencite[23]{Sutcliffe14}.
The more entrenched the ``world religion'' model becomes,
the more difficult it becomes to `conceptualize a different paradigm
of classification of the data for ``religion''\thinspace'
\parencite[25]{Sutcliffe14}.

Carol Cusack grapples with a similar challenge in \textit{Invented Religions}.
She recognises a pattern in the ``world religions'' that their origin is
either --- in the case of Judaism, Christianity and Islam ---
associated with divine revelation or --- in the case or Hinduism ---
`so far in the past that individual founders are unknown but venerability is
assured' \parencite[1]{Cusack10}.
In a similar vein, she observes in the case of new religions that
scripture is usually said to originate from
an authoritative external source like God, and the associated teachings
are argued to be `not really ``new'' but rather a contemporary
strand of ancient wisdom' \parencite[1]{Cusack10}.
Invented religions --- as defined by Cusack --- reject this
``web of conventions'', however, through their unapologetic affirmation of
human origin.
Their beliefs might be structured around the content of a novel,
as in the case of the Church of All Worlds (53),
or perhaps they emerged as a product of late-night discussion in
24-hour bowling alleys, in the case of Discordianism (28).
In spite of this `deeply provocative' rejection,
however, Cusack suggests that invented religions `can be seen to be
functionally similar, if not identical, to traditional religions'
when examined through the lens of Peter Berger and Thomas Luckmann's
social-constructionist model of world-building, and she further reinforces
the possibility of their religious status through the adoption of a polythetic
definition of religion. These will be explained in turn.

Berger and Luckmann \parencite*[79]{Berger67} wrote in
\textit{The Social Construction of Reality} that
`Society is a human product. Society is an objective reality.
Man is a social product'.
Berger later applied that dialectic model of society to religion
in \textit{The Social Reality of Religion},
and it is the summary he gives of the theory
in the latter book which will be reproduced here.
The dialectic model of society suggests that
`the two statements, that society is the product of man
[\textit{sic}] and that
man is the product of society, are not contradictory'
\parencite[3]{Berger69}.
Put another way, Berger (3) observes that although
`there can be no social reality apart from man',
`it is within society, and as a result of social processes,
that the individual becomes a person,
that he attains and holds onto an identity,
and that he carries out the various projects that constitute his life'.

This dialectic relationship between humans and society is said to operate
through a three-step process of
`externalization, objectivation, and internalization' (3--4).
Externalization is the moment in which humans create society
through the `ongoing outpouring' of physical and mental activity
into the world (4).
The human origin of externalization's product is obscured
in the moment of objectivation, however,
resulting in its objective appearance (4).
Finally, it is through internalization that the external
reality of society influences humans, resulting in
the socialisation of the individual
`\textit{to be} a designated person
and to \textit{inhabit} a designated world'
and thus providing a `meaningful order, or nomos' (16, 19; emphasis in original).
The role played by religion in this world-building process
is that of ``cosmization'', that is,
`the identification of [the] humanly meaningful world [nomos]
with the world as such, [...] reflecting it or being derived from it
in its fundamental structures' (28).
In other words, religion is a process through which humans
come to take society for granted as part of ``the nature of things''
--- the cosmos --- in order to stabilise their socially-constructed reality (25).
In this way, `human order is projected into the totality of being'
and `the entire universe [is conceived of] as being humanly significant' (28).

Religion, understood through this framework as a vehicle for cosmization,
`is, to a large extent, about narrative and the success of the story'
and, in Cusack's mind,
the stories offered by invented religions function in much the same way
as those offered by the ``world religions'' \parencite[4]{Cusack10}.
To suggest, then, that invented religions are not \textit{really} religions,
would be to claim that they lack some essential quality held by
the apparently functionally similar world religions
which benefit from undisputed religious status.
Cusack traces this line of reasoning back to essentialist definitions
of religion which are underpinned by reasoning convincingly
reproduced by Sam Gill \parencite*[968]{Gill94}:
\begin{displayquote}
  To study Christianity [...] is simply to study things Christian.
  Perhaps it seemed logical to extend this principle to the general
  academic study of religion by arguing that the academic study of religion
  is the study of data that are distinctively and uniquely religious.
  A definition of the essence of religion would function for the academic
  study of religion, it might be supposed, something like doctrine or a
  statement of faith.
\end{displayquote}
The trouble with such a definition is not only its exclusivity but the
obscurity of its mechanism.
To investigate the nature of religion requires analysis of religions,
and religions are identified by recognition of their religious essence.
In Gill's (968) words, `the unreachable goal towards which the study
is directed, that is to understand what religion is,
is required as a precondition to the study'.
In order to avoid such an impasse --- and the aforementioned exclusivity ---
Cusack adopts a polythetic definition.

Polythetic definitions operate on the assumption that
`no single distinguishing feature, or no specific conjunction
of distinguishing features, can universally be found in what,
on various grounds, we may wish to identify as ``religions''\thinspace'
\parencite[158]{Saler93}.
Instead, a ``family resemblance'' model, `preeminently associated with
Ludwig Wittgenstein' \parencite[159]{Saler93}, is employed which unites
religions `based on a set of characteristics, only some of which a system
must have in order to be counted as a religion' \parencite[158]{Wilson98}.
Cusack \parencite*[20]{Cusack10} gives an exemplory list
of nine characteristics commonly found among religions including
`belief in supernatural beings, [and] notions of sacredness and profanity',
suggesting that if all were present then `it is probable that what is being
observed is a religion', but if only some were present then `further
evidence of the movement's religious nature might be required'.
In this manner, a spectrum is formed with
`no sure or stable border where religion ends and
nonreligion begins' \parencite[396]{Saler99}.
Boundary cases like TempleOS, with challenging mixtures of
religious and non-religious characteristics, can then be analysed in order
to clarify what religion ``looks like''.

TempleOS' value as a boundary case can be seen in its encapsulated rejection
of a number of intuitive binary distinctions.
It challenges the online--offline distinction, for example, through
its combination of presence on the internet in the TempleOS.org website
and forums of discussion, and significant orientation toward the
focal-point of Davis' personal ``offline'' condition and day-to-day life.
As highlighted in the competing streams of reception in the previous chapter,
TempleOS also problematises the sacred--secular binary,
with a significant portion of its audience perceiving and evaluating it
as an impressive feat in software engineering while others suggest
that its oracle offers legitimate divination and that Davis'
purported worldview elucidated profound truths.
The boundary transgression that will be considered at length for the remainder
of this chapter, however, is that of the mechanical and spiritual.

Erik Davis' \parencite*[2--3]{Davis98} \textit{TechGnosis} explores this
juxtaposition of the mechanical and spiritual in depth,
going no further than the introduction without recognising the intuitive
juxtaposition of technology and the mystical:
\begin{displayquote}
  Common sense tells us that mysticism has no more in common with technology
  than the twilight cry of wild swans has with
  the clatter of Rock'em Sock'em Robots. [...]
  According to this narrative, technology has helped disenchant the world,
  forcing the ancestral symbolic networks of the old to give way to the crisp,
  secular game plans of economic development, skeptical inquiry,
  and material progress.
\end{displayquote}
He aims with his book to present a different view of modern technology,
suggesting that it `embodies an image of the soul, or rather a host of images:
redemptive, demonic, magical, transcendent, hypnotic, alive'
\parencite[9]{Davis98}, the possibility of which is aptly captured in
science-fiction writer Arthur C. Clarke's \parencite*[21]{Clarke73} Third Law
that `any sufficiently advanced technology is indistinguishable from magic'.
Davis provides two explanations of Clarke's Third Law, the first being
sceptical and negative, and the second being mystical and positive.


Recognising Clarke as `a rationalist (if an often mystical one)',
Erik Davis first puts forward his negative explanation, that
`what [Clarke] seems to mean is that, in sociocultural terms, advanced
technologies \textit{appear} to be magical' (180; emphasis in original)
and goes on to highlight the degree to which this law applies particularly
to digital technology with the image of a broken-down car:
\begin{displayquote}
  Twenty years ago, you had half a chance of fixing your car; these days,
  with computer chips and miniature sensors scattered through the vehicle
  [...] you need some serious tech just to hack the nature of a glitch.
  The logic of technology has become invisible --- literally, occult.
  Without the code, you're mystified.
  And nobody has all the codes anymore.
\end{displayquote}
A useful connection might be drawn here with TempleOS' use of a high-speed
stopwatch in its oracle for divination.
A simplified version of the program might work with a list of ten words
and a counter incrementing once per second from the number one to the number
ten; when a button is pressed, the counter's current state would be used to
select the corresponding word from the numbered list.
If the numbered list of words and the state of the counter were visible,
the user could quite easily time their pressing of the button to select
a word of their choosing.
Through complexifying and obscuring the counter, however, an apparently
random result can be achieved through a totally predictable,
mechanical operation.
In both the case of the stopwatch and the digital components of the car,
the logic of the technology is invisible in the sense that
it is sufficiently complex and obscured that the fine details of
its function are impercievable without
the use of a digital tool for inspection.

This point can also be usefully extended to explain
the importance that Terry A. Davis placed
on having written the HolyC compiler himself.
A compiler is a program which `translates a whole program, called the
source code, at once into machine language before the program is executed'
\parencite[323]{Computers}.
Machine language is `the only programming language that the computer can
understand', but `is made up entirely of 1s and 0s'
\parencite[321]{Computers}.
A compiler is useful, then, because it enables programmers to write
source code in high-level languages --- like Terry A. Davis' adaptation of the C
programming language, HolyC --- which `closely resemble human language'
\parencite[323]{Computers}.
In a phone call received at 10:49 a.m. during a live stream
on February 6, 2017, Terry A. Davis  forcefully expressed his opinion that
`the difference between an amateur and a professional is you write your
own compiler, OK? I have a 20,000-line divine-intellect compiler'
\parencite[31:44]{Telephone}.
One of Terry A. Davis' intentions for TempleOS was
``recreational programming'', involving the reading, writing,
and compiling of source code.
Thus, although the source code of a program is typically invisible
to its users, in the context of TempleOS the further degree of obscurity
provided by the compiler is meaningful.
Programs written in source code only become functional
once they have been compiled into machine code, and although the TempleOS'
compiler is available for anybody to use, Terry A. Davis was the programmer
who developed it.
In other words, it is his compiler which performs the ``magical'' process
of translating the inert source code into the powerful
machine language --- the visible into the invisible.
Terry A. Davis, as ``the high-priest of God's official temple'',
holds a unique insight into the ``occult'' process performed by the compiler.

Reinforcing this connection between computers and magic,
the anthropologist T. M. Luhrmann \parencite*[106]{Luhrmann89},
writing about modern witchcraft in England, observes that
`perhaps one or two out of every ten magicians [she] met
had something to do with computers'.
Indeed, some particularly relevant evidence for this can be found
in the signature appended by Richard Mitton to his earlier-mentioned blog post
``A Constructive Look at TempleOS'' in which he indentifies himself as a
`software engineer and travelling wizard' \parencite{CodersNotes}.
In an effort to explain this observation, Luhrmann suggests that
`both magic and computer science involve creating a world
defined by chosen rules, and playing within their limits' and that
`both in magic and in computer science words and symbols have a power which
most secular, modern endeavors deny them' \parencite[106]{Luhrmann89}.
This observation is explicitly drawn upon by Erik Davis
\parencite*[181]{Davis98}, who goes on to suggest that the world-building
power of both magic and computer science appeals for its production of
`the illusion of leading the mind ever closer to its longed-for
mastery of matter' and supposes that if we accept
`that appearances compose our world as much as truths,
then the ceaseless emergence of advanced technologies that define life
on the flying crest of the twenty-first century may paradoxically draw us into
a silicon wizard world'.

In this manner, Erik Davis recognises similarity both in appearance
and in function between computers and magic.
He suggests that computers, by virtue of their technological complexity,
resemble magic in the invisibility of their operation and that both can
serve as tools --- for the programmer and magician, respectively ---
which work to build worlds.
Referring back to Cusack, Berger, and polythetic definition, we might find
in the ``occult'' qualities of computers and magic the religious family
characteristics of awe and faith.
Furthermore, we can see that both magic and programming can take a
central role in the task of world-building.
Erik Davis suggested that the motivation behind world building efforts
is a `longed-for mastery of matter', and Berger's conception of
world building was as an effort to project human order into the cosmos,
but both can be identified in TempleOS' oracle.
The oracle can be understood as, for Davis, at once a source of evidence
that the world is, in Berger's terms, humanly significant, and a material
project constructed in a pragmatic effort access that evidence.
TempleOS, in this way, powerfully evidences Erik Davis' argument that
computers, in spite of their origin in scientifically-driven advances in
technology, need not be understood in opposition to the mystical.


%who draws an analogy
%between the function of new technology and magic which both, in his words,
%`[open] up novel and and protean spaces of possibility within social reality
%[...] [allowing] humans to impress their dreaming wills upon
%the stuff of the world, reshaping it, at least in part,
%according to the designs of the imagination'.

%Erik Davis also draws a connection between magic, computer science,
%and science fiction.
%He refers to The Church of All Words (a group later recognised by Cusack as
%an ``invented religion''), suggesting that `allusions to science fiction
%and fantasy fiction are staples of hacker culture', one reason for which being
%that `science-fiction and fantasy writers don't just tell tales --- they build
%worlds' \parencite[182]{Davis98}.
%The suggestion here is that magic, fiction, and computer science all involve

%Although Cusack is concerend with world building and externalisation,
%Davis is concerned with

%(hacker culture?) Erik Davis \parencite*[166]{Davis98} links even Apple with spirituality, writing
%`With a name that hearkened back to Eden's fruit of knowledge
%(and an initial selling price of \$666), the Apple proffered the
%Promethean dream of putting godly power in your hands'.


\end{document}
