\documentclass{article}
\usepackage[authordate,backend=biber]{biblatex-chicago}
\addbibresource{diss.bib}

\usepackage{setspace}
\doublespacing
\usepackage[activate={true,nocompatibility},final,tracking=true,kerning=true,spacing=true,factor=1100,stretch=10,shrink=10]{microtype}

\usepackage{amssymb}

\author{David Mynors}
\title{Davis and NIST}
\date{2019-01-17}

\begin{document}
\section*{Divination}

Fritz Graf \parencite{Divination11} derives the term `divination'
from the Latin \textit{divinare}, ``to ascertain the divine will'',
and describes it as ``a procedure employed in many religious cultures
to discover by ritual means what is hidden from human knowledge''.
He goes on two explain that divination practices function
either directly, or indirectly.
The first category involves the concerned individual experiencing
special dreams or ecstasy--possibly against their will--and interpreting
those experiences in order to ascertain communication from a deity.
The latter category involves the interpretation of
``a number of arbitrary signs'' as a message from a deity.

The most prominent divination process employed by Davis involved
the United States National Institute of Science and Technology (NIST)
and falls into the latter of the aforementioned categories.
Davis explains the process in a video titled
\textit{TempleOS: NIST Random Beacon} (CITE?),
beginning with the example of state lottery websites.
``Every day they publish a new batch of random numbers
that were chosen by ping pong balls, so these are true random numbers,
they are not pseudorandom numbers''.
NIST however, as Davis subsequently introduces,
publish random numbers every minute using, in Davis' words,
``special hardware that's not pseudorandom numbers, [but] real random numbers''.
Davis takes the first five digits from the most recent number
published by the NIST randomness beacon,
and then looks up the corresponding line number from
a 100,000 line text document on his computer containing the King James Version.

\clearpage
\printbibliography
\end{document}
