\documentclass[Draft.tex]{subfiles}

\begin{document}
\chapter*{Introduction}
\addcontentsline{toc}{chapter}{Introduction}

The development of TempleOS is understood to have begun in 2003,
but it was not until 2013 that it was released under that name.
Its creator, Terry A. Davis, described it as ``God's official temple''
and understood himself to be its high priest.
He amassed a significant online following
in the years prior to his death in 2018
who would interact with him during his livestreams
and formed online communities
in which they continue to discuss his life and work.
TempleOS constitutes new data for the academic study of religion,
and this dissertation aims to investigate how it might prove valuable
in discourses surrounding the definition of religion
and its relationship with computers and the internet.

The first chapter aims to introduce readers to TempleOS and Davis.
It will begin with an exposition of the technical function
and complexity of TempleOS,
explaining that it is an operating system rather than an application,
and that it would typically be installed in a ``virtual machine''.
In order to contextualise the interest that it generated
in technical communities, an effort will be made to convey
its magnitude as a project undertaken by a single programmer.
Davis' offline biography will then be presented
with reference to sources widely cited by his followers.
In addition, an effort will be made to illustrate
the personality that Davis presented in his live streams.
Finally, an overview will be given of the initial reception of TempleOS
during Davis' life and the state of the project in the wake of his death.

The second chapter will explore the tension between
Davis' intention for TempleOS and its reception.
Reception theory will be invoked at the outset,
with reference to James Machor's entry on the topic in
\textit{Vocabulary for the Study of Religion}.
Evidence for Davis' intention will be drawn from
various pages on his website and videos that he uploaded to the internet.
Some attention will be paid to
the metamorphosis of the TempleOS project over time.
Competing streams of reception will be evidenced
by the reactions expressed in comments by ``accidental'' audiences on YouTube
and the discussion on the much more ``intentional'' community of r/TempleOS.

Finally, the third chapter will explore the analytic value of TempleOS
in various debates in Religious Studies.
It will begin with an examination of the definition of the religion,
assessing the coherence of TempleOS with categories like ``world religion''
and ``new religious movement''.
In this vein, Carol Cusack's Invented Religions will be introduced
to offer insight into the legitimation strategies of new religious movements
and the mode of their formation,
with reference to Peter Berger's social-constructionist
theory of religion as world-building.
The status of TempleOS will be considered here with reference to
Davis' external legitimation of revelation, the sincerity of his followers,
and its free-form structure of online organisation.
TempleOS' transgression of the online--offline, sacred--religious,
and mechanical--spiritual boundaries will then be expounded
through the presentation of scholarship addressing
the relationship of religion, computers, and the internet.
Erik Davis' Techgnosis will be utilised to explain
the mystifying ``occult'' qualities of computers, and to
elucidate the influence of ``hacker culture'' in online religious practice.
Following this, Heidi Campbell will be referenced to explore
more general characteristics of manifestations of religion on the internet,
and N. J. Demerath's work on re-sacralization will be used to
illuminate Davis' efforts to imbue the otherwise disenchanted
realm of algorithms with a new spiritual reality.

\end{document}
