\documentclass[Draft.tex]{subfiles}

\begin{document}
\chapter*{Introduction}
\addcontentsline{toc}{chapter}{Introduction}

The development of the Temple Operating System\footnotemark\thinspace
is understood to have begun in 2003,
but it was not until 2013 that it was released under that name.
Its creator, Terry A. Davis, described it as ``God's official temple''
and understood himself to be its high priest.
He amassed a significant online following
in the years prior to his death in 2018
who would interact with him during his live streams
and formed online communities
in which they continue to discuss his life and work.
TempleOS --- with its digital location and the ambiguity between
Davis' intention and its reception --- constitutes interesting new data
for the academic study of religion,
and this dissertation aims to investigate how
an examination of TempleOS might prove valuable
in discourses surrounding the definition of religion
and its relationship with computers and the internet.

\footnotetext{
  For the sake of simplicity, the Temple Operating System will be used
  to refer specifically to the operating system created by Terry A. Davis,
  and ``TempleOS'' will refer to the broader phenomenon of the operating system,
  its creator, and the associated online following.
}

The first chapter aims to introduce readers to Davis and ``the Temple
Operating System''.
It will begin with an exposition of the technical function
and complexity of the Temple Operating System,
explaining that it is an operating system rather than an application,
and that it would typically be installed in a ``virtual machine''.
In order to contextualise the interest that it generated
in digital-technical communities, an effort will be made to convey
its magnitude as a project undertaken by a single programmer.
Davis' biography will then be presented, including details about his
personal ``offline'' life, and his activities on the internet.

The second chapter will explore the tension between
Davis' intention for TempleOS and its reception.
Evidence for the inference of Davis' intention will be drawn from
various pages on his website and videos that he uploaded to the internet.
Competing streams of reception will then be evidenced
by the reactions expressed in comments by ``accidental'' audiences on YouTube ---
most of whom would not have encountered Davis before ---
compared with discussion amongst the much more ``intentional''
community of r/TempleOS\_Official, constituting of participants who had
actively sought out a space to discuss Davis and his work.

Finally, the third chapter will highlight the value of TempleOS for
boundary work in the definition of religion.
It will begin with an exposition of the purpose and function of boundary
work as a process demarcating distinct fields of knowledge.
The contemporary example of new age spiritualities will then be
introduced in order to highlight difficulties that can be encountered
with definitional boundaries --- in particular, the ``world religions
paradigm''.
The contemporary importance of boundary work will then be reinforced
through reference to the example of Carol Cusack's \textit{Invented Religions}.
Her use of Peter Berger and Thomas Luckmann's social-constructionist model of
world-building and adoption of a polythetic definition of religion will then
be explained, highlighting the possibility for a more fluid conception of
``religion''.
The chapter will culminate in a presentation of Erik Davis' argument in
\textit{TechGnosis} that modern technology can, and does, embody mystical
qualities.
Through this, it will be ultimately concluded that TempleOS, in its transgression
of the intuitive mechanical--spiritual distinction, constitutes valuable
new data which challenges traditional images of
computer science as a disenchanted realm of algorithms with no space for
spirituality.

\end{document}
