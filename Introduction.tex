\documentclass[Draft.tex]{subfiles}

\begin{document}
\chapter*{Introduction}
\addcontentsline{toc}{chapter}{Introduction}

The development of TempleOS is understood to have begun in 2003,
but it was not until 2013 that it was released under that name.
Its creator, Terry A. Davis, described it as `God's official temple'
and understood himself to be its high priest.
Davis amassed a following online in the years prior to his death in 2018 who
would interact with him during his livestreams and formed online
communities in which they continue to discuss his life and work.
TempleOS constitutes new data for the academic study of religion,
and this dissertation aims to investigate how it might prove valuable
in discourses surrounding the definition of religion
and the its relationship with computers and the internet.

%TempleOS is a liminal case, with opposing characteristics held in tension.
%It collapses the binary between secular and religious in its
%dual function as oracle and recreational programming environment,
%it injects charismatic divine inspiration into the otherwise
%spiritually barren domain of computation,
%and it problematises the distinction of online and offline as
%Davis amassed a following on the internet who took great interest
%in his offline situation.
%Moreover, TempleOS was shared and discussed on the internet,
%but the operating system itself lacked any networking capability,
%necessitating its use in a more personal, offline capacity.

The first chapter aims to introduce readers to TempleOS and Davis.
It will begin with an exposition of the technical function and complexity
of TempleOS, explaining that it is an operating system
rather than an application,
and that it would typically be installed in a ``virtual machine''.
In order to contextualise the interest that it generated
in technical communities, an effort will be made to convey
its magnitude as a project to be undertaken by a single programmer.

The second chapter will explore the tension between
Davis' intention for TempleOS, and its reception.
Reception theory will be invoked at the outset,
with reference to James Machor's entry on the topic in
\textit{Vocabulary for the Study of Religion}.
Evidence for Davis' intention will be drawn from various
pages on his website and videos that he uploaded to the internet.
Some attention will be paid to the metamorphasis of the TempleOS project
over time. The reception of TempleOS will be evidenced by the reactions
expressed in comments by ``accidental'' audiences on YouTube
and discussion on the much more ``intentional'' community on r/TempleOS.

Finally, the third chapter will frame TempleOS in the context of various
debates in Religious Studies.
Namely, it will consider whether TempleOS might be considered a ``religion''
at all, referencing Carol Cusack's \textit{Invented Religions},
Peter Berger's social-constructionist theory of religion as world-building,
and Demerath's work on secularisation and re-sacralization.
Following that analysis, TempleOS will be considered in light of the discourse
surrounding the relationship between religion, computers, and the internet.
Particular use will be made here of Erik Davis' \textit{Techgnosis} and
the work of Carol Cusack on online religion.

\end{document}
