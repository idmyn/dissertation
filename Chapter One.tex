\documentclass[a4paper]{article}
\usepackage[utf8]{inputenc}

\usepackage[authordate,backend=biber]{biblatex-chicago}
\addbibresource{diss.bib}
\DeclareLabeldate{\field{date}\field{eventdate} \field{origdate}\literal{nodate}}

\usepackage[activate={true,nocompatibility},final,tracking=true,kerning=true,spacing=true,factor=1100,stretch=10,shrink=10,letterspace=50]{microtype}
\microtypecontext{spacing=nonfrench}
\usepackage[T1]{fontenc}
\usepackage[bitstream-charter]{mathdesign}
\usepackage[scaled]{beramono}
\usepackage{setspace}
\setstretch{1.29}
\usepackage{parskip}

\setlength{\skip\footins}{20pt}

\usepackage{titlesec}
%\titleformat*{\section}{\large\bfseries}
\titleformat*{\section}{\Large\scshape\lsstyle}
\titlespacing*{\section}
{0pt}{7pt}{0pt}



\author{David Mynors}
\title{What is TempleOS?}

\begin{document}
\section*{The operating system}

TempleOS is an operating system (OS) written by Terry A. Davis.
Its development is said to have begun in 2003
and it was first released with the name `SparrowOS' in 2012
before it was renamed to TempleOS the following year.
Its reported purpose varies depending on where you look.
In the introduction section of its documentation, it describes itself as
an ``operating system for recreational programming'' most beneficial for
``professionals doing hobby projects'', ``teenagers doing projects'',
and ``non-professional, older-persons projects''.
In its charter, however, it describes itself as ``God's official temple''
and proclaims that
``just like Solomon’s temple, [it] is a community focal point
where offerings are made and God’s oracle is consulted''
\parencite{Charter}\footnotemark.
In this vein, one can press the F7 key anywhere in the OS
to access an oracle which produces words on the screen,
and Shift-F7 for random bible passages.
Indeed, in addition to the OS offering
a malleable environment for recreational programming,
it contains games, Moses comics, and hymns authored by Davis.

\footnotetext{In-text citations without dates will typically
reference `live' websites as they appear at the date of access.
In the case of archived pages, the date on which they were archived
will be used as date of publication.
In both cases the date of publication listed on the page itself
will be referenced, if one can be found.}

\section*{Davis' life}
The `cult following' of TempleOS has perhaps
more to do with Davis himself than his creation, however.
The principal source of bibliographical information about him is
an article titled \textit{God’s Lonely Programmer}
in VICE Magazine’s \textit{Motherboard} publication.
In it, Jesse Hicks \parencite*{Hicks14}---drawing
upon ``two months of emails and phone conversations''---first
introduces the OS,
but spends the majority of the article de-mystifying the background of Davis.
According to the article,
Davis was born in 1969 in Wisconsin and grew up Catholic.
He started using computers in elementary school
and went on to complete bachelor's and master's degrees
in electrical engineering at Arizona State University.
In his blog (01/23/17), Davis writes that
between 1990 and 1996 he was ``as atheist as they come'',
and is quoted in Hicks' article saying
``I thought the brain was a computer [...] so I had no need for a soul''.
He goes on to say, however, that he is differentiated from other atheists
because ``God has talked to me,
so I'm basically like an atheist who God has talked to''.
Davis describes starting to
``[see] people following me around in suits and stuff'' in mid-March 1996
and later ends up in a mental hospital for two weeks
following a run-in with the police.
His mental state stabilises later that year
and he moves back in with his parents,
proceeding to experience manic episodes every six months for the next six years.
The article reports that he hadn't been to the hospital since then, however,
with Davis claiming
``for those first few years, I was genuinely pretty crazy in a way.
Now I'm not. I'm crazy in a different way maybe'',
and that he's ``learned not to freak out''.
For biographical information beyond this date,
the unofficial `Timeline of TempleOS' written by Issa Rice
\parencite*{Rice18} can be consulted.
The timeline 39 entries of bibliographical information
about Davis' `personal' life, and a near-equal 40 entries about
the online activities of Davis from 2001 until 2017.
These online activities included his creation---and
often the subsequent suspension---of accounts on various platforms,
and his notable forum posts, video uploads, and live streams.
After living homeless for approximately one year following
his eviction from his family home by his parents,
Davis was hit by a train on the 11th of August 2018 and died \parencite{Cecil18}.

\section*{Davis' online following}
Davis' following can be found across platforms like YouTube and Reddit,
and engage in discussion (serious and humorous) about
technical aspects of the OS, and Davis’ personal life.

\printbibliography
\end{document}
