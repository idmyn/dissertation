\documentclass[Draft.tex]{subfiles}

\begin{document}
\chapter{An Introduction to `Invented Religions'}

\section*{Introduction}
The objective of this chapter is to introduce
the `invented religions' category
and exposit the parallels between religion and play,
culminating in an overview of Discordianism.

\section*{Exposition}
Normally, new religions employ various strategies
to defuse the otherwise troubling unfamiliarity
that might be felt by people encountering them for the first time.
Carol Cusack \parencite*[1]{Cusack10} refers to this as the
``web of conventions that surround the establishment of new religions'',
and notes that it typically involves
``arguing that the teaching is not really `new' but rather
a contemporary strand of ancient wisdom'',
or suggesting that scripture originates from
an authoritative external source, like God.
`Invented religions' are those which decline to employ these tactics,
instead choosing to openly declare their invented status.
Their beliefs might be structured around the content of a novel,
as in the case of the Church of All Worlds (53),
or perhaps they emerged as a product of late-night discussion in
24-hour bowling alleys, in the case of Discordianism (28).

Cusack (1) invites us to see something ``deeply provocative''
about this invitation for us to recognise as `religion'
these movements which present themselves unashamedly as
products of the human imagination.
And yet, she argues, ``they can be seen to be functionally similar,
if not identical, to traditional religions''
and ought thus to be taken seriously by scholars.
Indeed, when asked in an interview for the Religious Studies project
why scholars should take invented religions seriously,
she  gestured towards ``Peter Berger and Thomas Luckmann’s
social-constructionist model of reality-building'' in order to suggest
that ``there is a sense in which every religion has been invented''
\parencite*{CusackPodcast}.

Berger and Luckmann \parencite*[79]{Berger67} wrote in
\textit{The Social Construction of Reality} that
``Society is a human product. Society is an objective reality.
Man is a social product''.
Berger later applied that dialectic model of society to religion
in \textit{The Social Reality of Religion},
and it is the summary he gives of the the theory
in the latter book which will be reproduced here.
The dialectic model of society suggests that
``the two statements, that society is the product of man
[\textit{sic}] and that
man is the product of society, are not contradictory''
\parencite[3]{Berger69}.
Put another way, Berger (3) observes that although
``there can be no social reality apart from man'',
``it is within society, and as a result of social processes,
that the individual becomes a person,
that he attains and holds onto an identity,
and that he carries out the various projects that constitute his life''.
\textbf{This could be clearer?}

This dialectic relationship between humans and society is said to operate
through a three-step process of
``externalization, objectivation, and internalization'' (3--4).
Externalization is the moment in which humans create society
through the ``ongoing outpouring'' of physical and mental activity
into the world (4).
The human origin of externalization's product is obscured
in the moment of objectivation, however,
resulting in its objective appearance (4).
Finally, it is through internalization that the external
reality of society influences humans, resulting in
the socialisation of the individual
``\textit{to be} a designated person
and to \textit{inhabit} a designated world''
and thus providing a ``meaningful order, or nomos'' (16, 19).
The role played by religion in this world-building process
is that of `cosmization', that is,
``the identification of [the] humanly meaningful world [nomos]
with the world as such, [...] reflecting it it or being derived from it
in its fundamental structures'' (28).
In other words, religion is a process through which humans
come to take society for granted as part of `the nature of things'
--- the cosmos --- in order to stabilise their socially-constructed reality (25).
In this way, ``human order is projected into the totality of being''
and ``the entire universe [is conceived of] as being humanly significant'' (28).

Religion, understood through this framework as a vehicle for cosmization,
``is, to a large extent, about narrative
and the success of the story'' \parencite[4]{Cusack10}.
With this in mind, and in light of Cusack's (4) observation
that the most resonant stories of the twentieth century
fell broadly in the science fiction genre,
one can see how a religion framed by the cosmology of
the \textit{Star Wars} films, for example, might emerge.
\textbf{Polythetic definition.}



Furthermore, these movements inflame some of
the broadest questions asked by the academic study of religion,
sometimes even voicing these questions themselves.
In response to their listing on various websites as a `parody religion'
in 2001, Discordians began an email protest with one writer demanding
``that either you move us into the same category
as the rest of the religions, or tell me what the criteria [are]
to become a `real' religion so that I might show
how Discordianism meets them'' \parencite[209]{Chidester05}.

We've covered the \textit{how}, now let's consider the \textit{why}.
Cusack \parencite*[7]{Cusack10} argues that ``invented religions are
exercises of the imagination that have developed in a creative
(though sometimes oppositional) partnership with the influential
popular cultural narratives of the contemporary West'',
and that this would not have been possible without
the development of secularisation, individualism, and consumer culture
``from the late nineteenth century to the present''.
The ``retreat of institutional Christianity'' opened the door
for new religions to flourish and synthesise a `spiritual supermarket',
offering the opportunity to ``select beliefs and practices,
often from different traditions, that were not only incompatible
but in some cases diametrically opposed'' \parencite[25, 17]{Cusack10}.
Members of these new religious forms are less likely to ask
`is it true?', and more likely to ask `does it work?'.
Cusack \parencite[9]{Cusack10} points to Yves Lambert's observation that
these new religious movements, in their ``reinterpretation and innovation'',
commonly exhibited the characteristics of
``worldliness, dehierachization of the human and the divine,
self-spirituality, [and] parascientificity'' \parencite[303]{Lambert99}.


\section*{Play}
With reference in particular to Discordianism,
which will be discussed at the end of this chapter,
Cusack \parencite*[23]{Cusack10} notes a particular presence of
``persuasive discourse and artistic creativity'' and links it to
Lonnie Kliever's 1981 article which concludes that
``artful argument and artistic vision are noncoercive and nonviolent ways
of fashioning consentual communities'' with the power to
``construe the universe `as if' it were humanely ordered and meaningful,
even though we know these construals are
`not really' true'' \parencite[665]{Kliever81}.
The crux of the argument in Kliever's article is that
in both play and religion we
``contest and escape reality \textit{knowingly}'' (662),
that is, we `remember while forgetting' (660).

Religious belief-systems, Kliever argues,
are ``essentially fictive in character'' (658).
Put another way, they are
``symbolic constructs which \textit{cannot} be verified
and hence \textit{cannot be true}'' (658).
Contrastingly, `Facts' are
``symbolic constructions which have been established as
reliable representations of a world that actually does exist
independently of all human imagination and intervention'' (658).
This unverifiability does not diminish the value of religion,
however, as ``scientific knowledge and bureaucratic order
have not eradicated or satisfied human craving
for life-enhancing and death-defying symbols and rituals'' (658).
Humans have a need for `something more' than
the material reality before us, so the argument goes,
which only fictions can offer it.
Kliever recognises the `inherent irony' in the proposition that
``illusion can triumph over reality for a time
even while it is known to be illusion'' (659),
but invites critics to consider the powerful analogy of play.

To introduce the concept of `remembering while forgetting'
in the context of play, Kliever points to the work of
Gregory Bateson (1955 in Kliever 1981, 661),
who developed his theory upon an observation he made at a zoo.
Bateson observed two monkeys play-fighting and realised that
this phenomenon of apparent combat with both parties lacking
the intention to harm the other was only possible
``with some degree of \textit{meta-communication}'',
essentially conveying the message
``These actions in which we now engage do not denote
what those actions \textit{for which they stand} would denote'' (661).
In other words, to physically push somebody in jest,
one has to communicate that the action performed
is \textit{not} an invitation for a fight, as it might typically suggest.
Communicating this message involves reference to
a scenario that exists only hypothetically and
performing an action that, taken at face value, is untrue (662).
For a playful shove to be succeed, however,
the recipient should receive it in good spirit, that is,
with an `unthinking' recognition of its playfulness.
The recipient must assess the intention behind the action sufficiently
to realise that it does not denote an invitation to fight,
but not so much as to completely defuse the action
and lose sight of the playful provocation entirely.
Put in Bateson's psychological terminology,
one must balance the \textit{primary} process of thought which
avoids ``distinctions [...] between language and reality'',
and the \textit{secondary} process of thought
which ``discriminat[es] between literal and figurative''
(Bateson in Kliever 1981, 662).
It is along these lines that Kliever argues that we can
employ religion as a primary process in tension with
the secondary process that criticises it as fiction.

\textit{Consider `remembering while forgetting' in Discordianism?}

\section*{Discordianism}
Cusack (27) introduces Discordianism as
``a religion devoted to Eris, the Greek goddess of chaos''
which spread, after its founding in 1957, ``through
the novels of Robert Anton Wilson [...] in the 1970s,
and through role-playing game clubs, science fiction fandom,
and the internet from the 1980s to the present''.
The religious `tradition' of writing scripture was upheld in Discordianism,
she notes, with the ``anarchic 'zine'' titled \textit{Principia Discordia}.
In light of their scripture's lack of sustained narrative, their teachings
are ``complex and difficult to expound in an ordered fashion'' (29),
but include some internally consistent philosophical ideas
alongside some comically absurd statements.
For example, the \textit{Principia Discordia} in one place introduces the
`psycho-metaphysical' `Aneristic Principle of apparent order',
explaining that ``ideas-about-reality are mistakenly labeled `reality'
and unenlightened people are forever perplexed by the fact that
other people, especially other cultures, see `reality' differently.
It is only the ideas-about-reality which differ'' \parencite[49]{Principia}.
One of its five commandments --- named `the Pentabarf' ---
on the other hand, insists that
``A Discordian shall Partake of No Hot Dog Buns'' (4).


\end{document}
