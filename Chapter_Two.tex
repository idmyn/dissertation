                                                                                %
\documentclass[Draft.tex]{subfiles}

\begin{document}
\chapter{An Introduction to `Invented Religions'}

\section*{Introduction}
The objective of this chapter is to introduce the possibility of
`the sacred` being found in unexpected places.
First, an overview of Carol Cusack's vision of
the `invented religions' category will be provided.
Then, an exposition will be given of her example of Discordianism
which will segue into Kliever's discussion of religion and play
before finally a summary will be given of Demerath's observation
that `the sacred' might be found in non-religious places.

%provide an overview
%of the scholarship surrounding the `invented religions' category,
%beginning with Carol Cusack's \parencite*{Cusack10} vision, and
%moving on to consider the interlinked topic of non-religious
%sources of the sacred with reference to N. J. Demerath \parencite*{Demerath00}.

\section*{Exposition}
Normally, new religions employ various strategies
to defuse the otherwise troubling unfamiliarity
that might be felt by people encountering them for the first time.
Cusack refers to this as the
``web of conventions that surround the establishment of new religions'',
and notes that it typically involves
``arguing that the teaching is not really `new' but rather
a contemporary strand of ancient wisdom'',
or suggesting that scripture originates from
an authoritative external source, like God (1).
`Invented religions' are those which decline to employ these tactics,
instead choosing to openly declare their invented status.
This might be because they their beliefs are structured around
the content of a novel,
as in the case of the Church of All Worlds (53),
or because they emerged as a product of late-night discussion in
24-hour bowling alleys, in the case of Discordianism (28).

Cusack (1) invites us to see something ``deeply provocative''
about this invitation for us to recognise as `religion'
these movements which present themselves unashamedly as
products of the human imagination and yet, she argues,
``they can be seen to be functionally similar, if not identical, 
to traditional religions''
and ought thus to be taken seriously by scholars.
Furthermore, these movements inflame some of
the broadest questions asked by the academic study of religion,
sometimes even voicing these questions themselves.
In response to their listing on various websites as a `parody religion'
in 2001, Discordians began an email protest with one writer demanding
``that either you move us into the same category
as the rest of the religions, or tell me what the criteria [are]
to become a `real` religion so that I might show how Discordianism meets them''
\parencite[209]{Chidester05}.

We've covered the \textit{how}, now let's consider the \textit{why}.
Cusack \parencite*[7]{Cusack10} argues that ``invented religions are
exercises of the imagination that have developed in a creative
(though sometimes oppositional) partnership with the influential
popular cultural narratives of the contemporary West'',
and that this would not have been possible without
the development of secularisation, individualism, and consumer culture
``from the late nineteenth century to the present''.
Secularisation meant that  ``throughout the twentieth century
religious institutions had lost much of their influence'',
however there was also ``growth in new religious movements
in the post-war period'' (9).
Cusack \parencite[9]{Cusack10} points to Yves Lambert's observation that
these new religious movements, in their ``reinterpretation and innovation'', 
commonly exhibited the characteristics of
``worldliness, dehierachization of the human and the divine,
self-spirituality, [and] parascientificity'' \parencite[303]{Lambert99}.


% \section*{Alternative sources of the sacred}
% An obviously allied concept to that of invented religion
% is the idea that `the sacred' might be sourced
% from places other than religion.
% N. J. Demerath \parencite*{Demerath00} gives a compelling presentation
% of this, arguing \textit{blah}.





\end{document}