\documentclass[Draft.tex]{subfiles}

\begin{document}
\chapter{An Analysis of the Intention Behind and Reception of TempleOS}

The objective of this chapter is to explore dymanic tension between Davis'
intention for TempleOS and its reception in two online audiences ---
YouTube commentors and participants in r/TempleOS\_Official.
Davis' technical ambitions for the operating system were covered in the
previous chapter.
Here, he will be shown to have understood his production of TempleOS
as part of a wider desire reform Christianity.
The reception of TempleOS will be shown to be mixed, however,
with some expressing sympathy for Davis as a talented programmer who
struggled with mental health difficulties while others suggesting that
his worldview ought to be taken seriously.

\section*{Intention}

\begin{displayquote}
  I am Roman Catholic, and the religion for my kingdom
  would be Roman Catholic, but the church is dead [...].
  Right now the computer industry is sick like Herod,
  the Roman Church is sick like Herod,
  the world is sick like Herod, and I'm going to cure it.
	(Terry A. Davis, \textit{I'm Starting a New Religion} [Internet Archive, 2018])
\end{displayquote}

TempleOS is introduced at the beginning of its charter as
`God's official temple. [...] [A] community focal point
where offerings are made and God's oracle is consulted' \parencite{Charter}.
The remainder of the document, however, is dedicated to technical details,
such as the `limit of 100,000 lines of code for all time'
and `just one 8x8 fixed-width font'.
Further explanation of its religious purpose is located somewhat sporadically
on other pages on the TempleOS website, and in videos uploaded by Davis.
In \textit{The 64-Bit Operating System}, for example,
Davis \parencite*{64-Bit} introduced TempleOS
with reference to its oracle feature --- which utilises a high-speed
stopwatch operated by the user to select random words and bible passages ---
suggesting that its `main purpose is for doing offerings of hymns and art
and poems and stuff and then getting a response from God in the oracle'.

On a page titled \textit{New Religion}, and under the heading
``Christianity, a Charity or a Church?'', Davis \parencite*{NewReligion}
wrote that `today, Christianity is a secular humanist social club
that does charity.  A church should primarily love God and do prayers'.
Further down, under the heading ``The Counter-Renaissance'',
Davis described his
\begin{displayquote}
	`dream that obsession with God in the United States
	will return to the level it was in Europe in 1200 A.D.
	Europeans built cathedrals and had monastaries
	that wrote beautifully decorated books. [...]
	people will strive to make God's temple beautiful, glorious
	and as perfect as possible.  It will be adored.
	People will do offerings in God's temple and God will talk'.
\end{displayquote}
On another page titled \textit{The Purpose of Life},
Davis \parencite*{PurposeLife} commanded `You don't know God. [...]
You must talk with God to know Him. [...] Seek the Lord by taking initiative',
and proceeded to explain his rationale as follows:
\begin{displayquote}
	There's something obviously different about people in the Bible
	compared to people today --- God talked!
	Also, the people in the Bible were obsessed with doing offerings all the time.
	It is required that you do offerings before God will talk.
	Did the people in the Bible hear voices?  Maybe.
	More likely, they used occult techniques such as an oracle.
\end{displayquote}
Furthermore, Davis \parencite*{Demands} understood himself to have a role
more significant than that of an ordinary software developer,
introducing himself on his \textit{Demands} page as
`high priest of God's official temple, TempleOS' with
`divine authority to command any company in the computer industry
to do anything that [he] deem necessary to make God's temple
more beautiful, glorious and perfect'.
He understood his obligations in this role to be twofold: he was
`in charge of the core 100,000 lines of TempleOS code'
and he did `continual offerings to keep God entertained'.
It appears, then, that Davis wanted to, in some sense, reform Christianity
with a renewed focus on communication with God
as it is presented in the Old Testament ---
by giving offerings and consulting an oracle.


\section*{Reception}
The most-viewed video included in search results on YouTube
for the query ``Terry Davis TempleOS'' is
\textit{TempleOS | Down the Rabbit Hole} (henceforth \textit{DRH})
by Fredrik Knudsen with 1.6 million views
and a running time of one hour and 25 minutes.
The next highest viewcount is \textit{Internet Insanity: Terry A. Davis}
(henceforth \textit{Insanity}) with 761 thousand views and a running time of 17 minutes.
Both of these videos were uploaded as episodes of ongoing series
--- ``Down the Rabbit Hole'' and ``Internet Insanity'', respectively ---
focussed on controversial or eccentric online personalities
to YouTube channels with hundreds of thousands of subscribers.
If these two videos are excluded from the first ten videos appearing
in the search results for ``Terry Davis TempleOS'', the mean view-count
for the remaining eight stands at roughly 100 thousand.
It can be inferred from this disparity in view count that a significant
proportion of the viewers of \textit{DRH} and \textit{Insanity}
had not come accross Davis before.
\textit{Insanity} was uploaded to YouTube in mid-2017 and adopts a damning stance
towards Davis' mental illness, introducing him as
`a man who had a psychotic break.
A talented man, but [one] who encountered mental illness so severe
that any legacy he has will be overshadowed
by the sideshow that he has become' \parencite*{Metokur17}.
Contrastingly, Knudsen's \parencite*{Knudsen18} video
was uploaded a few months after Davis' death and
takes a more balanced approach, introducing TempleOS as,
`depending on who is consulted, [...] the outdated product of a deranged mind,
the work of a misunderstood genius, or some complicated combination of the two'.

YouTube Commentor Hey Trey (2019) aptly summarises
the narrative arc of Knudsen's portrayal of Davis:
\begin{displayquote}
	i started the video really disliking the guy,
	thinking to myself what a smug pos [piece of shit].
	transitiones [\textit{sic}] into, this guy is crazy and racist, and comedy gold.
	Then, as the video went on i went from laughing, to absolute total pity
	and a feeling of total sadness that he was very obviously
	becoming very unhealthy and mentally unstable.\footnotemark
\end{displayquote}
Perhaps resulting from this dramatic narrative arc in Knudsen's portrayal of Davis,
the comments on this video --- when sorted by ``top comments''
to prioritise those given the most ``thumbs up'' ---
are overwhelmingly sympathetic towards Davis, with many expressing similar
sentiments to Hey Trey.

\footnotetext{
	\textit{The Chicago Manual of Style} \parencite*[15.51]{CMOS}
	suggests that blog comments should be referenced in-text
	with an indication of their date of publishing
	but should not appear in the bibiography. That style will be adopted here.
}

Sympathetic comments can also be found on \textit{Insanity} --- particularly
since Davis' death --- but the less sympathetic tone of the video itself
results in more varied comments,
which I have grouped into four broad themes of sympathy, humour,
admiration of talent, and worldview evaluation.
Comments referencing Davis' talent often mention his mental illness.
Gaming with Mikey! (2018) wrote `The dude built an OS from scratch?
That's genius level ability. It's just tragic that he's crippled by his
mental disability', and
Teamugi (2018) wrote `Schizophrenia aside, this guy has talent.
Building an OS from scratch by yourself [...] is extremely impressive'.
Many of the humour-oriented comments reference computing ---
Mayor of Gaming (2018) wrote `As a programmer, I'm positive that
a few of my colleagues are 1 compile error away from becoming this guy'.
Similarly, one sees references to popular culture in comments suggesting
that Davis' purported worldview was illuminating.
For example, Darkwing Dumpling (2018) wrote
`Terry is the one who coded the matrix[.] We need him',
and a comment from Heinrich Nornelius Agrippa (2018)
reproduces a message received by players in the videogame
\textit{The Elder Scrolls III: Morrowind} if they kill an
``essential character'' in a quest:
`with this character's death,
the thread of prophecy is severed.
Restore a saved game to restore the weave of fate,
or persist in the doomed world you have created'.
There are also some sincere comments imploring readers to
consider Davis' worldview seriously.
\foreignlanguage{russian}{братишка} (2018), for example, wrote
`if you watch his ramblings long enough,
they actually start to make sense. im serious. try it'.

The comments left by viewers on YouTube can be usefully contrasted
with threads found on the TempleOS subreddit\footnotemark.
Notably, the highest rated post of all time on r/TempleOS\_Official
is a link to Knudsen's \textit{TempleOS | Down the Rabbit Hole} video,
posted by Klyke \parencite*{Klyke18},
and some of the comments on the post highlight the presence of
an intentional and sincere community.
BiggRanger wrote (on December 22, 2018)
`seems like a pretty well done video [...].  The guy did his homework on this,
but still has the feel of an outsider looking in'.
Another commenter, GDP10, wrote (on December 24, 2018)
`I don't really like how at some points this video portrays Terry as a madman.
It's not really fair.  Many of his moments of ``insanity''
are actually quite lucid and he has sound points. [...]
Rest in peace Terry.  I know you can finally talk with God face to face'.
This comment is of particular interest because it not only explicitly validates
Davis' worldview, but it later includes a quote from a different post on
r/TempleOS\_Official made four months prior and,
in a follow-up comment replying to Cu\_de\_cachorro, GDP10 also references
a post made five years prior by Davis \parencite*{Davis13}
about TempleOS on r/programming.
GDP10's first quote discourages the reader from fixating on the perceivable
symptoms of Davis' illness but to consider the extent to which it caused him
suffering, and the second is an invitation for the members of r/programming
to `tolerate or ignore his condition
while exploring what he's spent a lot of time on'
(ba-cawk commenting on 21 March, 2013).
These quotes, included alongside hyperlinks to their sources and with added bold and italics,
are indicative of the extent to which members of r/TempleOS\_Official
are able to refer back to often quite obscure content on the internet produced
by, or about, Davis, and the care and effort which is frequently employed
in doing so.
%One particularly notable example of this is a post made by z0mbiee
%\parencite*{z0mbiee18}, titled
%\textit{Looking for a specific Terry video where he calls the puppy warrior dog}.
%The post reads:
%\begin{displayquote}
%  `back when he [Davis] was living with his parents
%  he made a video of him in his bedroom, lying on the bed with the pup,
%  calling him warrior dog and I can't find the video anywhere [...]
%  I've gone through so much of his archived videos
%  that I'm losing the will to live trying to find it'
%\end{displayquote}
%Within twentry-four hours, bailbondsh\_ commented with a link to the video,
%which was stored on


A more recent post by MicroPeanor \parencite*{MicroPeanor19} highlights
both the depth and religiosity of discussion had on r/TempleOS\_Official.
The title asks
\textit{Is TOS a hidden, misunderstood, technological gem
or more of a fun little thing to tinker with?}
and the body of the post indicates the author's specific interest in
TempleOS' technical value:
\begin{displayquote}
	I think when TOS [TempleOS] first came around people were focused
	on the wrong aspects of it.  [...]
	The creator had [...] over saturated his work with god [...].
	People looking into it now how a different outlook on it is seems.
	They’re finding that it is potentially useful and far more advanced
	than people originally thought.
\end{displayquote}
The comments include discussion of TempleOS' technical merit and religious
content with roughly equal weighting.
The two most highly rated comments directly answer the question asked by
MicroPeanor in the post's title --- the second doing so in six paragraphs
with five hyperlinks --- but the third answers
`Both. Terry was a fascinating guy too.
Most of yall are probably atheists but I think
he did follow God's will in a weird sort of way'
(redditpostingM223540 on 21 February 2019).
MicroPeanor's response (on the same day) to this comment
draws heavily on langauge of ``seekership'' and draws a parallel between
Davis and biblical figures who were treated with scepticism:
\begin{displayquote}
	Nah actually not atheist. Although I don’t follow one specific religion.
	I feel many different religions have something to offer,
	so taking a bit of moral guidelines from the ones I more closely agree with
	is what I do.  [...]
	He very well could have been following God’s will.
	There a a [\textit{sic}] crap load of stories in the Bible
	where god talks to or through someone and everyone around them doubts them.
	I can’t say he was or wasn’t being talked to,
	but he sure thought he was doing the right thing.
\end{displayquote}
A\_Plagiarize\_Zest also responded (on 1 March, 2019) to redditpostingM223540,
writing `I think hes [Davis is] correct in thinking God
is the random synchronicity of the universe' and draws on anecdotal evidence
to support his view:
\begin{displayquote}
	`Sometimes he would get response from god that were hilarious or perfectly fit.
	One time he was saying how depressed he was and so he ``asked god''
	what he should do, and the random generator responded
	with ``plant trees'' [...].
	I think terry was really talkin to god. I cant tell you how many times
	Ive been researchin something on youtube, or gone down some rabbit hole,
	and then Ill go on 4chan or watch some podcast and the exact same shit
	I randomly started researchin gets discussed on the podcast.
	Its like god sayin, ``you're on to somethin.''
\end{displayquote}
Furthermore, when Hastaroth (the following day) dismissed
the latter anecdotal evidence as `Nothing special or magical [...].
It's called the Baader-Meinhof effect', A\_Plagiarize\_Zest responded
(the same day) with three substantial paragraphs dismissing Baader and Meinhof
as `typical ``intellectuals'' that base intelligence off of
how well they can subvert the human race
with bullshit philosophy and bullshit science'.
Other comments blend the technical with the religious.
WPLibrar2 wrote (on 2 March, 2019):
\begin{displayquote}
	The entire OS is one of the best system existing simply because
	as a temple it is meant to be a technological masterpiece
	and a giant offering, increasing the accuracy
	of talking to god simply by that.
	This is literally how all religions, abrahamatic [\textit{sic}] and pagan
	and all temples, worked in history.
	And god the way he did that generator...
	I have never seen something as spirited as this one.
	What I mean is that through the whole system supporting the generator,
	he put the actual concept of god into a simple machine.
	TempleOS is the only system having something called ``spirit''.
\end{displayquote}

There is a definite sense, then, of competing streams in the reception
of TempleOS.
Audiences on YouTube, for many of whom it would have been their first
encounter with Davis, reacted to the story of Davis' life and work
with admiration and sympathy.
Members of the r/TempleOS\_Official community, however, can display
great sincerety in the effort they expend to properly format their posts,
even including hyperlinks to related posts from months before.

\footnotetext{
	Subreddits are community pages formed around topics of interest on
	Reddit.com and are denoted by the ``r/'' prefix --- the
	TempleOS subreddit, for example, is named r/TempleOS\_Official.
	Reddit's homepage features posts from some of the most popular
	subreddits by default --- r/books, for example ---
	and the ``default subreddit'' /all is a compilation of the most popular
	posts from a wide variety of subreddits measured by subtracting their
	``downvotes'' from their ``upvotes'',
	but posts on the vast majority of the roughly 1.2 million subreddits
	are only seen by those who visit them directly.
	For scale, r/science, a default subreddit to which all new accounts
	are subscribed, has roughly 18 million subscribers,
	r/soccer 850 thousand, and r/TempleOS\_Official 1,800.
}

\end{document}
