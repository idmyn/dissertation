\documentclass[Draft.tex]{subfiles}

\begin{document}
\chapter*{Final Thoughts}
\addcontentsline{toc}{chapter}{Final Thoughts}

TempleOS is a potent illustration of the fertility of digital space
for religious formations.
This dissertation has introduced Terry A. Davis, his creation, and their
following as an invitation for boundary work addressing the intersection
of religion with new technology.

Terry A. Davis, as a self-described `atheist who God has talked to',
encapsulates in his biography and in the avowed aims of the
Temple Operating System a transgression of
the sacred--secular binary \parencite{Atheist}.
Similarly, The Temple Operating System exhibits concurrently the characteristics
of transparent mechanism --- through the accessibility of its source code ---
and thorough spirituality --- through its omnipresent oracle function, available
in any location in the operating system at the press of a key.
Indeed, the community surrounding Terry A. Davis and his creation show a sincere
reverence toward Terry A. Davis both as a technically gifted programmer and as a
mystical pioneer, and the functioning of magic and programming alike in human
efforts of world-building has been shown, through Erik Davis' \textit{TechGnosis},
to support the possibility of a wider dovetailing of the mystical with the digital.

\end{document}
